\documentclass[../main/main.tex]{subfiles}
\begin{document}

%\addparttoc{Introduction générale}
\chapter*{Introduction générale}\label{cp:intro}

Cette thèse s'inscrit dans le cadre de la cosmologie observationnelle,
c'est à dire l'étude de l'Univers par le biais d'observations.

Aujourd'hui, et grâce notamment à la théorie de la relativité générale
d'Albert Einstein, un modèle cosmologique décrivant admirablement bien
de nombreuses observations indépendantes a vu le jour. Ce modèle, dit de
concordance, décrit un Univers plat, ayant la particularité d'être
en expansion accélérée et dont le contenu est dominé par deux
composantes encore incomprises: la matière sombre et l'énergie sombre.
Leur nom réfère directement à l'impossibilité de les observer
directement, mais seulement à partir des effets dont elles sont la
cause.
Comprendre leur nature et leurs propriétés est devenu un des enjeux majeurs de la cosmologie
moderne.

L'énergie sombre est par exemple la cause principale de l'accélération
de l'expansion de l'Univers, et ses propriétés peuvent être étudiées en mesurant les
distances d'objets lointain dans l'Univers. On utilise pour cela ce que
l'on appelle des chandelles standards, objets astrnomiques dont la
luminosité intrinsèque est connue \textit{a priori}. En comparant cette
luminosité à celle observée, il est alors possible de remonter à la
distance de ces sources lumineuses.

Les supernovae de type Ia (SNeIa), explosion particulière d'une étoile,
possèdent ces propriétés et peuvent ainsi être utilisées comme indicateurs de distances. C'est en utilisant cette sonde
cosmologique que l'accélération de l'expansion de l'Univers a été mise en évidence en 1998 par
\citet{Riess1998,Perlmutter1999}. Leur luminosité
pouvant dépasser celle de leur galaxie hôte, elles permettent en plus de mesurer
des distances très lointaines dans l'Univers. Ces détections nous
permettent alors
d'étudier l'évolution de la dynamique de l'Univers de nos jours jusqu'à
plusieurs milliards d'années dans le passé.

À l'heure actuelle, les incertitudes sur les caractéristiques de
l'énergie sombre sont dominées à parts égales entre le nombre de SNeIa dans les
échantillons (incertitude statistique) et les incertitudes
systématiques.

La nouvelle génération de relevés astronomiques de SNeIa va
drastiquement diminuer les erreurs statistiques, avec de nouvelles
méthodes de relevés automatisés sur tout le ciel observable et à haute
cadence. À bas redshift par exemple (Univers proche pour $0<z<0.1$), le relevé Zwicky
Transient Facility (ZTF) a déjà détecté près de $3000$ SNeIa sur le point
d'être révélées publiquement, alors
que la plus grande compilation à ce jour pour cette profondeur ne
dépasse pas $400$ SNeIa.
La statistique sera tout autant multipliée à plus grande distance,
notamment avec le Large Synoptic Survey Telescope (LSST) entre
$0.2<z<1$.

Les efforts futurs pour une étude précise de l'énergie sombre vont donc progressivement
se diriger vers les incertitudes systématiques. L'une d'entres elles
provient de la contamination des échantillons de SNeIa par d'autres
types de supernovae, n'ayant pas les caractéristiques de chandelles
standards, et pouvant induire des biais importants dans la détermination
de paramètres cosmologiques. Bien que des méthodes de classification photométrique
existent, le moyen le plus précis de classifier une supernova
observée repose sur l'analyse de son spectre. En effet, chaque type de
supernova présente des caractéristiques spectrales qui lui sont propres,
et cette information n'est pas accessible en photométrie.

Cependant, cette méthode de classification n'est actuellement possible que
lorsque la SN est suffisament isolée de sa galaxie hôte. En effet, plus
l'explosion survient proche du coeur de la galaxie, plus la
contamination spctrale s'intensifie et la classification devient
difficile voire impossible. L'objectif de ce travail de recherche est de
répondre à cette problématique dans le cadre du relevé astronomique ZTF,
avec l'utilisation d'un spectrographe 3D dédié à la classification des
SNe.

Nous commencerons dans la première partie de ce manuscrit par introduire
le contexte scientifique général, en introduisant quelques notions
cosmologiques importantes et l'utilisation des supernovae comme sonde
cosmologiques. Nous présenterons ensuite la collaboration et le relevé
ZTF, en nous concentrant sur la détection photométrique des
SNeIa. L'instrument que nous utilisons est un spectrographe 3D, utilisé
conjointement aux détections photométriques et appartenant à la
collaboration. Un chapitre sera ainsi dédié aux
concepts de la spectrographie 3D et à la présentation de l'instrument
utilisé. Nous détaillerons la méthode d'extraction de spectre de SNe
actuelle et ses limites en cas de contamination par la galaxie hôte.

La deuxième partie de cette thèse introduira la preuve de concept d'une
nouvelle méthode d'extraction permettant de lever la contamination
entre la SN et sa galaxie hôte, sous la forme d'un outil de modélisation
de scène. Nous proposons pour cela d'utiliser des
données photométriques de la galaxie obtenues en amont de l'explosion de
la SN. En utilisant les connaissances de la physique
des galaxies, nous modélisons à l’échelle locale les propriétés
spectrales de l'hôte afin de créer un cube 3D modèle de la galaxie
isolée.

\end{document}

%%% Local Variables:
%%% mode: latex
%%% TeX-master: t
%%% End:
