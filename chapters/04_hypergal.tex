\documentclass[../main/main.tex]{subfiles}
\begin{document}
%\dominitoc
%\faketableofcontents
\setcounter{chapter}{4}
\chapter{\hypergal: Modéliseur de scène pour l'extraction de sources ponctuelles}\label{ch:hypergal}

\minitoc
\vspace{2cm}
La première partie de ce manuscrit était dédiée à la présentation du
contexte scientifique dans lequel ce travail de recherche est
effectué.

Nous avons dans un premier temps introduit les notions de
cosmologies nécessaires pour comprendre l'environnement scientifique de
travail, ainsi que la nature et le rôle des supernovae de type Ia en
tant que sondes cosmologiques.

Dans un second temps nous avons présenté la collaboration Zwicky
Transient Facility, ses différents groupes de recherches et plus
particulièrement la place qu'occupe l'étude des SNeIa dans ce relevé
astronomique nouvelle génération. 
Après avoir introduit la nécessité d'une méthode de
classification spectroscopique des évènements transitoires détectés par
la caméra ZTF, nous avons présenté la Spectral Energy Distribution
machine, un spectrographe 3D que possède la collaboration et
conçu pour la classification.

Le pipeline de réduction de données actuel, \pysedm, permet également une
extraction des sources ponctuelles observées par la caméra de l'IFU de
la SEDm. La méthode implémentée est toutefois rudimentaire, et ne permet
pas de palier aux nombreuses situations de contamination de la source
ponctuelle par sa galaxie hôte.

Non seulement cela induit une perte statistique de supernovae
classifiables non négligeable, mais de surcroît cela induit un biais
environnemental dans l'échantillon des SNeIa de ZTF.

C'est pour répondre à cette problématique que nous introduisons
\hypergal, un modéliseur de scène pour l'extraction de sources ponctuelles.

\newpage

\section{Idée générale}\label{sec:ideahypergal}

\subsection{Problématique}

Le champ de vue de la SEDm étant étroit ($28\arcsec\times28\arcsec$), nous
avons en général $3$ composantes qui composent la
scène, à savoir le fond du ciel, la galaxie hôte et la source ponctuelle.

La difficulté majeure d'une modélisation de scène hyperspectrale (3D) provient de la chromaticité de chacune de ces
composantes, et plus particulièrement de la galaxie qui est une source
structurée de forme et chromaticité variable.

Une première idée serait d'attendre l'atténuation de l'évènement
transitoire, réobserver l'hôte, et projeter cette seconde acquisition
dans l'espace de la première observation afin d'isoler la source
ponctuelle \citep{Bongard2011}. Une telle approche est envisageable pour une extraction de
quelques cibles, mais en aucun cas à notre époque où les relevés grands
champs et à haute cadence deviennent légion et observent des milliers
de supernovae par an.

Le but d'\hypergal\ est de pouvoir modéliser la scène observée
par la SEDm après réduction des données, c'est à dire le cube 3D,
directement après l'observation.

Il va donc non seulement falloir trouver un moyen de modéliser chacune
des composantes, mais également de les projeter dans l'espace des
observations de la SEDm. Cela implique une étude approfondie des caractéristiques de
l'instrument mais également de prendre en compte les conditions
d'observation comme l'atmosphère le long de la ligne de visée.


\subsection{La composante galactique}

La motivation principale de ce modéliseur de scène est le fait que nous
avons des informations sur la galaxie hôte \underline{avant}
l'apparition de l'évènement transitoire. En effet, d'autres relevés
astronomiques comme le Sloan Digital Sky Survey (SDSS;
\citet{YorkSDSS2000}) ou Panstarrs \citep{ChambersPanstarrs} ont couvert
des portions de ciel communes avec ZTF, et permettent donc de remonter à
des informations photométriques de la galaxie encore exempte de la supernova.

Souhaitant une modélisation spectrale de la galaxie, il faut donc un
moyen de passer de l'espace photométrique à l'espace hyperspectral, autrement
dit estimer la distribution spectrale en énergie (\textit{Spectral Energy Distribution}, SED) de la galaxie. 

L'objectif est de pouvoir recréer un cube
3D contenant uniquement la galaxie et de le projeter dans
l'espace de la SEDm, en prenant en compte les propriétés de l'instrument et les
conditions d'observation. 
La modélisation doit donc se faire localement pour
que chaque spaxel du cube modèle ait son propre spectre associé.

Une approche triviale de ce problème serait de partir d'images de
plusieurs bandes
photométriques de la galaxie, et interpoler un spectre pour chaque pixel
de ces images à l'aide par exemple d'un
simple polynome. Cela permettrait de créer un cube 3D \textit{ad hoc}
avec une source spatialement structurée.
Mais grâce à l'avènement de nombreux instruments terrestres et spatiaux
lancés au cours des dernières décennies, nous avons une certaine
connaissance de la composition d'une galaxie, et ainsi des différentes
contributions qui forment sa SED.

L'idée est donc d'ajuster la distribution spectrale en énergie avec les données
photométriques de la galaxie, afin de construire un cube 3D qui servira de base pour le modéliseur de scène.

\section{Ajustement de la distribution spectrale en énergie}\label{sec:sedfitting}
\subsection{SED d'une galaxie}

La distribution spectrale en énergie est, par définition, l'évolution de
l'énergie émise par un objet en fonction de la longueur d'onde, à la
différence d'un spectre où on trace le flux ou la densité de flux.

Avant d'utiliser un ajusteur de SED nous allons aborder
quelques connaissances physiques existantes du spectre d'une galaxie.

Des rayons $\gamma$ au domaine radio, la SED d'une galaxie est définie par sa composition en matière
baryonique, et de leurs intéractions physiques complexes. En mettant de côté la matière
sombre qui n'intéragit pas avec le champ électromagnétique, une galaxie
est principalement composée d'étoiles de tout âge, de gaz atomiques
moléculaires ou ionisés et de poussières. Les étoiles
qui composent la galaxies (entre $10^{8}$ et $10^{14}$) émettent la
lumière qui nous permet de la détecter. Le gaz interstellaire
et la poussière vont quant à eux principalement altérer la SED: le gaz en ajoutant des raies
d'émission et d'absorption, la poussière en provocant une atténuation par
absorption et diffusion des radiations dans l'UV jusqu'au proche infrarouge, puis en
ré-émettant cette énergie dans l'infrarouge moyen/lointain.

L'état de ces composantes et
leurs intéractions nous renseignent sur les propriétés physiques fondamentales
de la galaxie: le taux de formation stellaire (SFR) et son histoire
(SFH), la masse stellaire, la metallicité, les propriétés d'atténuation, la masse de
poussière, les émissions nébulaires ou encore la présence possible d'un noyau actif (AGN). 

La SED d'une galaxie contient ainsi l'empreinte de tous ces ingrédients
et phénomènes physiques complexes, évoluants au cours du temps et traçant
l'histoire de la galaxie. Deux exemples de spectres de galaxies obtenus
avec le relevé SDSS dans l'optique sont
présentés dans la Figure~\ref{fig:specgalsdss}.

\begin{figure}
  \centering
  \includegraphics[width=0.99\textwidth]{../figures/04_hypergal/specgalsdss.png}
  \caption[Exemple de spectres de galaxies]{Exemple de spectres de
    galaxies (crédit SDSS). \textit{À gauche}, le spectre d'une galaxie
    spirale avec présence d'une forte raie d'émission
    $\text{H}\alpha$. Cela se traduit par une forte présence d'étoiles
  jeunes (bleues) et de gaz qui favorise la formation stellaire. \textit{À droite},
le spectre d'une galaxie elliptique. On peut remarquer d'une part
l'abscence de raie d'émission $\text{H}\alpha$, d'autre part une
forte perte en flux vers $4000$\AA. Cela trahit la très faible présence
d'étoiles jeunes (bleues) dans la galaxie et un faible taux de formation
stellaire.}
  \label{fig:specgalsdss}
\end{figure}

Modéliser une SED galactique revient donc à comprendre chacune de ces
interactions et leur répercussions.

Malgré tout, certaines corrélations entre plusieurs paramètres rendent cette tâche
très difficile, comme par exemple la dégénerescence entre l'âge et la
métallicité \citep{Worthey}, ou encore l'âge et l'atténuation
\citep{Papovich}.

Ces deux dernières décennies ont été extrêmement riches en développement de modèles
et observations panchromatiques, permettant une compréhension de plus en
plus fine de la formation et l'évolution d'une galaxie.

Ont vu ainsi le jour des modèles de populations stellaires grâce à
\citet{Fioc1997}, \citet{BruzualCharlot2003} et \citet{Maraston2005}.
D'un autre côté, différentes lois d'atténuation par la poussière ont été
développées, comme par \citet{Calzetti1994,Calzetti2000} via l'étude de SED de
galaxies proches ayant un fort taux de formation stellaire, ou encore
avec une approche plus théorique de modèles de transferts radiatifs
\citep{WittGordon}.
Comme mentionné précédemment, la poussière ré-émet dans l'infrarouge et
l'étude et la modélisation de ce phénomène est un domaine actif de
recherche \citep{CharyElbaz, Draine2007,Casey2012, Dale2014, Leja2017}.

La manipulation de modèles pour chaque processus physique en oeuvre dans
une galaxie a permis l'émergence de nombreuses méthodes pour ajuster une
SED. Ces nouvelles techniques permettent ainsi d'inférer les propriétés intrinsèques des galaxies
observées (locales, globales ou les deux), de pouvoir interpoler un
spectre à partir d'informations photométriques ou encore d'en estimer le
redshift.

\subsection{Ajusteur de SED}

L'ajustement d'une distribution énergétique spectrale d'une galaxie est la
méthode première permettant d'inférer ses propriétés physiques intrinsèques à
partir d'observations photométriques. Ces propriétés peuvent ensuite être confrontés
aux prédictions provenant de théories d'évolution et formations de
galaxies. De ce fait, l'utilisation d'un SED Fitter est une pratique très
fréquente
lorsqu'il s'agit de tester des hypothèses en astronomie extragalactique
\citep{Tinsley1980, Walcher2011, Conroy2013, Chevallard2016, Briday22}. 

Trois composantes sont nécessaires pour procéder à un ajustement de SED : un
modèle physique qui décrit les différentes contributions qui la
composent, des données d'observations de la galaxie (photométriques
et/ou spectroscopiques) et l'ajusteur lui-même qui va inférer la
combinaison adéquate entre les modèles physiques et les observations.

De nombreuses techniques de SED Fitting ont été développées, certaines
basées sur 
la simple optimisation de vraisemblance, parfois appelée code
d'inversion, comme dans \pkg{ULySS}\footnote{\url{http://ulyss.univ-lyon1.fr}}\citep{KolevaUlyss}, \pkg{FIREFLY}\footnote{\url{http://www.icg.port.ac.uk/firefly/}}
\citep{WilkinsonFirefly} ou \pkg{LEPHARE}\footnote{\url{https://www.cfht.hawaii.edu/~arnouts/LEPHARE/lephare.html}} \citep{ArnoutsLephare} plus
axé sur la détermination de redshift photométrique.

Cette technique est très populaire de par sa rapidité de calcul et une
certaine simplicité à mettre en place. Néanmoins ces avantages sont
bridés par certaines limites. Par exemple un léger changement
dans les données d'entrées (comme un bruit dans une image photométrique
de galaxie) peut conduire à de grands écarts dans les paramètres
inférés \citep{Ocvirk}.
Par ailleurs, une méthode de maximum de vraisemblance peut être difficile à adapter
à des modèles hautement non-linéaires comme l'émission par la poussière.

Dans l'optique de résoudre ces problèmes, des techniques d'inférence
bayésienne ont à leur tour été développées. Avec cette méthode, des
grilles de paramètres sont pré-calculées puis comparées aux
observations. Le calcul de vraisemblance est alors très rapidement
déterminé, malgré le fait que le nombre de modèle à calculer au
préalable croît exponentiellement à mesure que l'on rajoute des
paramètres. Parmi les codes développés à partir de cette méthode, on peut citer \citet{Kauffman2003}, \citet{Salim2007}, le
framework \pkg{CIGALE}\footnote{\url{https://cigale.lam.fr}}
\citep{Burgarella2005, Noll2009, Boquien2019} ou encore
\pkg{MAGPHYS}\footnote{\url{http://www.iap.fr/magphys/}}
\citep{Cunha2008MAGPHYS}.

Cette approche, de par son succès, a rapidement été adoptée, et
étendu à un couplage avec des algorithmes de Monte-Carlo par chaînes de
Markov (MCMC) pour plus efficacement explorer l'espace des
posterior. Cette extension, initiée par \citet{Acquaviva2011} avec
\pkg{GalMC} (retiré du domaine public par faut de maintenance), puis
rapidement suivi de codes plus récents tels que
\pkg{BEAGLE}\footnote{\url{http://www.jacopochevallard.org/beagle/}}
\citep{Chevallard2016},
\pkg{BAGPIPES}\footnote{\url{https://github.com/ACCarnall/bagpipes}}
\citep{Carnall2018, Carnall2019} ou encore plus récemment
\pkg{Prospector}\footnote{\url{https://github.com/bd-j/prospector}}
\citep{JohnsonProspector} et
\pkg{piXedfit}\footnote{\url{https://github.com/aabdurrouf/piXedfit}}
\citep{Abdurro'ufPixedfit}.

Nous terminerons la présentation des SED Fitters en mentionnant le site
\url{sedfitting.org}, maintenu par Tamas Budavari, Daniel Dale, Brent
Groves et Jakob Walcher qui regroupe la grande majorité des codes et bases
de modèles disponibles publiquement.

\section{Présentation générale du Pipeline}\label{sec:pipeline}
%\label{sec:xxx}

\begin{figure}
  \centering
  \includegraphics[width=0.99\textwidth]{../figures/04_hypergal/softdaghypergal.pdf}
  \caption[Présentation du pipeline \hypergal]{Présentation du pipeline \hypergal.}
  \label{fig:softdaghypergal}
\end{figure}

Nous allons à présent introduire le modéliseur de scène \hypergal.
Les étapes principales de ce pipeline sont présentées dans la
Figure~\ref{fig:softdaghypergal}, et traceront l'organisation de cette
Partie du manuscrit.

\subsection{Cube intrinsèque}

Comme abordé dans la section~\ref{sec:ideahypergal}, le coeur
d'\hypergal\ repose sur la conception d'un cube 3D contenant la galaxie
hôte isolée de sa supernova: c'est la modélisation hyperspectrale de la galaxie. Le but n'est pas de remonter aux propriétés
intrinsèques de la galaxie, mais de simplement être en mesure
d'interpoler un spectre cohérent à l'échelle locale.

Cette étape, entièrement indépendante des observations de la
SEDm, va nécessiter l'utilisation d'un SED Fitter, que nous avons
introduit dans la section~\ref{sec:sedfitting}. Dans un premier temps,
nous allons 
récupérer des images de différentes bandes photométriques de la galaxie hôte de la supernova
détectée par ZTF. On procèdera ensuite à un fitting de SEDs de portions
locales de la galaxie, ce qui permettra d'obtenir une multitude de
spectres propre à chaque région de la galaxie. Avec un échantillonnage spectral
adéquat, nous serons ainsi en mesure de reconstruire un cube 3D, dont
les deux dimensions spatiales ($x$, $y$) seront définis par les images
photométriques, et la dimension spectrale par le SED Fitter. Le cube
résultant ne contiendra ainsi que la galaxie hôte, et sera appelé dans
la suite de ce manuscrit \textit{cube intrinsèque}. Cette étape de
modélisation hyperspectrale est détaillée dans le Chapitre~\ref{ch:modelhyperspec}.

\subsection{Modélisation de scène 2D}

Dans cette seconde étape, les observations de la SEDm deviennent
nécessaires: le but ici est de projeter le cube intrinsèque de la
galaxie dans l'espace des observations. Pour faire cela, nous allons de
façon indépendante caractériser la réponse impulsionnelle spatiale et
spectrale de la SEDm (Chapitre~\ref{ch:irf}).

En utilisant ces informations, nous projetterons dans l'espace de la
SEDm le cube intrinsèque préalablement scindé en $N$ méta-tranches
(2D). Il faudra pour cela prendre en compte la forme et la taille de l'échantillonnage
spatial des deux espaces (source photométrique et MLA de la SEDm) ainsi que la différence de seeing.
En plus de la composante galactique, nous caractériserons les composantes
de supernova, de fond de ciel et de potentiels artefacts à modéliser
pour compléter la scène.
La projection de chaque meta-tranche dans l'espace SEDm sera ajustée aux
meta-tranches correspondantes de l'observation, dont la minimisation permettra de récupérer un jeu de
$N\times2D$ paramètres.

\subsection{Modélisation chromatique et projection 3D}

Les $N\times2D$ paramètres sont ensuite utilisés pour étudier et fixer la chromaticité des composantes de la scène, comme la réponse impulsionnelle
spatiale de la SEDm (fonction d'étalement de point; PSF) ou la
variation de la position des objets dans le MLA due à la réfraction de
la lumière par l'atmosphère (ADR). Les modèles de chromaticité sont
déterminés \textit{a priori}, et les paramètres de ces modèles sont ajustés à
partir des $N\times2D$ paramètres obtenus de l'étape précédente.

Une fois les chromaticités fixées, l'ensemble des paramètres de projection de chaque tranche
du cube intrinsèque dans l'espace SEDm devient connue, et seuls les
paramètres d'amplitudes (fond de ciel, supernova...) sont ajustés pour
chaque longueur d'onde. Cette étape permet ainsi d'extraire les trois
composantes de la scène d'observation de la SEDm, à savoir le fond, la galaxie hôte et la
source ponctuelle.

\section{Cas pédagogique de présentation}

Les chapitres suivants de cette partie du manuscrit seront consacrés à
la description détaillée des différentes étapes du pipeline. Pour une illustration
appropriée, nous utiliserons un cas pédagogique de modélisation de scène
où la supernova est suffisamment éloignée de sa galaxie hôte ($\sim4\arcsec$).

La cible choisit est ZTF18accrorf, dont le cube extrait à partir d'une
observation de la SEDm est présenté dans la
Figure~\ref{fig:easycasesedm}.

\begin{figure}[ht]
  \centering
  \includegraphics[width=0.99\textwidth]{../figures/04_hypergal/e3dcube_ZTF18accrorf.png}
  \caption[Cube 3D SEDm de ZTF18accrorf]{Cube 3D d'une observation de
    ZTF18accrorf avec la SEDm. La figure de gauche montre les spectres
    en unité de pseudo-ADU, dont le code couleur correspond aux spaxels
    sélectionnés dans la figure de droite. La forme similaire des trois
    spectres malgré une localisation différente des spaxels sous-jacents dans le MLA est due à la
    présence du spectre du ciel sur l'ensemble du champ de vue. La croix rouge indique la
    position estimée de la supernova à partir des informations de
    guidage avec la Rainbow Camera. Dans ce cas ci, la supernova est
    assez aisément distingable de sa galaxie hôte, avec une séparation
    angulaire d'environ $4\arcsec$ et un redshift de $z=0.042$.}
  \label{fig:easycasesedm}
\end{figure}


\begin{figure}[ht]
  \centering
  \includegraphics[width=0.9\textwidth]{../figures/04_hypergal/easycaseextractionpysedm.png}
  \caption[Extraction de ZTF18accrorf avec \pysedm]{Extraction de
    ZTF18accrorf avec \pysedm. Sur la \textit{gauche} est représenté une image 2D
  du cube intégré spectralement, avec la croix rouge indiquant la
  position estimée de la supernova à partir des informations de
    guidage avec la Rainbow Camera. Les marqueurs noirs représentent les
  spaxels sélectionnés pour l'extraction automatique. \textit{En haut à
    droite} est représenté la modélisation d'une méta-tranche et le
  profil radial estimé. On voit clairement la composante de la galaxie
  hôte qui contamine l'extraction de la supernova. \textit{En bas} nous
  montrons en unité de flux calibré le spectre extrait par \pysedm en gris et le
meilleur modèle estimé par \pkg{SNID}. Même si certaines structures
semblent rappeler le spectre d'une SNIa, comme les absorptions FeII,
FeIII et MgII (entre $[4200-4600]$\AA, et entre $[4800-5200]$\AA), le
niveau de confiance pour la classification reste très faible, comme
l'atteste le $r\text{lap}=4.5$, et surtout le redshift estimé de
$z=0.178$ bien au dessus de la profondeur en magnitude atteignable par
la SEDm ($z_{lim}\sim0.1$).  }
  \label{fig:easycaseextractionpysedm}
\end{figure}


Malgré la localisation relativement excentrée de la source ponctuelle,
son faible contraste vis à vis de son hôte rend difficile son extraction
automatique par le pipeline \pysedm. Nous illustrons cette tentative
dans la Figure~\ref{fig:easycaseextractionpysedm}.

Après la présentation d'\hypergal\ à travers ce cas pédagogique, nous
montrerons un cas d'extraction extrême où la supernova est quasiment
confondue avec le coeur de sa galaxie hôte.


%\bibliographystyle{../main/aa_url}
%\bibliography{99_references}
\end{document}

%%% Local Variables:
%%% mode: latex
%%% TeX-master: t
%%% End:
