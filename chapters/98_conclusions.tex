\documentclass[../main/main.tex]{subfiles}
\begin{document}

\addparttoc{Conclusion générale}
\chapter*{Conclusion générale}\label{cp:conclusions}
\vspace{1cm}

Ce travail de recherche effectué sur deux annnées a permis le développement
d'un modéliseur de scène, \hypergal, afin de répondre à la
problématique de contamination des supernovae par leur galaxie hôte.

La première partie de ce manuscrit avait pour but d'introduire le
contexte scientifique de cette thèse. Nous avons commencé par présenter les
supernovae de type Ia, leurs propriétés de sonde cosmologique, leur rôle dans
l'étude l'énergie sombre et les conséquences d'une contamination d'un
échantillon par d'autres types de SNe. Nous avons par la suite introduit
le relevé astronomique ZTF ainsi que l'instrument que nous utilisons, le
spectrographe 3D SEDm.

Nous avons ensuite détaillé les étapes de conception d'\hypergal, en
introduisant la preuve de concept de cette nouvelle méthode d'extraction
de spectre de SNe.
Le coeur de ce pipeline repose sur l’utilisation de
données photométriques de la galaxie, prises en amont de
l’explosion de la SNIa. Grâce à l'utilisation du SED Fitter
\pkg{Cigale}, nous avons modélisé localement les propriétés spectrales
de la galaxie, permettant de créer un cube 3D modèle de l'hôte isolé de
la source ponctuelle.

Afin de pouvoir projeter ce cube intrinsèque dans l'espace des
observations de la SEDm, il a fallu caractériser la réponse
impulsionnelle de l'instrument. Avec l'observation de
lampes à arc, nous avons déterminé un modèle de
réponse impulsionnelle spectrale stationnaire, spatialement uniforme et
gaussien. La modélisation de sa chromaticité,
par un polynome quadratique,
a permis la projection du cube intrinsèque de la galaxie hôte dans
l'espace spectral de la SEDm.

Les supernovae étant des sources ponctuelles, nous avons également
caractérisé la réponse impulsionnelle
spatiale, nécessaire pour modéliser ces objets. Pour cela, un modèle analytique et empirique de fonction
d'étalement de point a été introduit, permettant de modéliser les effets
des turbulences atmosphériques sur les images astronomiques ainsi que des
effets instrumentaux. Entraîné sur des étoiles standards,
ce modèle est décrit par un profil radial asymétrique, possédant 2
paramètres de forme libres ajustés en
fonction des observations. En modélisant les chromaticités
intervenant sur ces paramètres et celle causée par la réfraction
différentielle atmosphérique, nous avons pu extraire le spectre de
plusieurs milliers d'étoiles standards. En les comparant aux spectres
spectrophotométriques de référence de calspec, cette étude a permis d'estimer
une précision de la calibration en couleur de l'ordre de $3\%$.

L'association du cube intrinsèque de la galaxie hôte et de la réponse
impulsionnelle de la SEDm a finalement permis la conception
du modéliseur de scène \hypergal, permettant
d'extraire séparément chaque composante de la scène. Une méthode de
classification automatique est également implémentée dans le pipeline,
en utilisant le classifieur de
SNe \pkg{SNID}.
Entièrement automatisé, \hypergal\ est optimisé
avec \pkg{DASK}, librairie de calculs parallèles qui commence à se faire
une place dans les analyses scientifiques lourdes. L'apprentissage de
cet outil a permis de drastiquement diminuer les temps de calcul,
passant de plusieurs heures en calcul linéaire à seulement une quinzaine de
minute, de la requête des données à l'obtention de tous les
résultats et figures de vérification. \hypergal\ peut ainsi être utilisé
aussi bien localement que sur un cluster distribué. 

Afin d'explorer les capacités du pipeline, nous avons conçu un jeu de
5000 simulations, basées sur l'observation de galaxies sans supernova
avec la SEDm. Les deux paramètres d'exploration de nos simulations sont
la distance entre la SN et le centre de la galaxie hôte, ainsi que le contraste
$c$, traçant l'intensité de la SN étudiée par rapport à celui de son
environnement. Nous avons extrait les SNe simulées avec \hypergal\
ainsi qu'avec une méthode d'extraction de référence, celle utilisée
préalablement par la collaboration sans modélisation hyperspectrale de
la galaxie. L'extraction de spectre avec \hypergal\ montre l'absence de corrélation entre la précision de
l'extraction et la distance SN-galaxie, nous permettant d'affirmer la capacité de
cette méthode à lever cette contamination. En
étudiant la distribution du contraste des observations réelles faites avec la SEDm, nous
avons également pu voir qu'\hypergal\ permettrait une classification
correcte de $95\%$ des SNeIa. Au delà d'un
contraste $c\sim0.2$ (ce qui représente plus de $90\%$ des observations), ce sont près de $99\%$ des SNeIa qui sont
correctement classifiées avec moins de $2\%$ de faux positifs,
témoignant de la confiance que l'on peut accorder à cette méthode
d'extraction de spectre.
Par comparaison avec la méthode de référence,
\hypergal\ permet de classifier correctement près de $20\%$ de SNeIa
supplémentaires entre $0.1<c<0.6$ (représentant $50\%$ des conditions
d'observation), avec seulement $\sim4\%$ de faux positifs dans cet
intervalle de contraste contre $\sim8\%$ pour l'autre méthode. Les
analyses relatives entre \hypergal\ et la méthode de référence sont
toutefois à prendre avec légèreté, étant donné que nous n'avons pas pris
en compte la distribution des distances SN-galaxies dans les données
réelles. Bien qu'\hypergal\ ne semble pas y être sensible, cela n'est
pas vrai pour la méthode de référence jusqu'à $\sim4\arcsec$ du coeur de
la galaxie.

Enfin, nous achevons ce travail de recherche avec la présentation de la
\textit{data release} 2 du groupe \textit{Type Ia Supernovae $\&$
  Cosmology}. Devant les améliorations significatives d'extraction
apportées par \hypergal\ il a été convenu que le spectre de chaque SN observée avec la
SEDm soit extrait avec la nouvelle méthode présentée dans cette
thèse. L'échantillon de SNeIa de cette DR2 est composé de $3792$ SNeIa, toutes classifiées
spectralement, et $65\%$ d'entres elles l'ont été
grâce à \hypergal. En appliquant certains critères de
qualité pour définir un \textit{golden sample} de $2905$ SNeIa (utilisé prochainement
pour l'estimation de paramètres cosmologiques), la contribution de la
SEDm et d'\hypergal\ reste de $65\%$. Finalement, il a été
montré qu'une coupure à un redshift $z<0.06$ permettrait de définir un
sous-échantillon exempt de toute fonction de sélection. Dans
ce volume limité de $1077$ SNeIa, $81\%$ d'entres elles ont été
observées par la SEDm et classifiées par \hypergal, illustrant la place
majeure qu'occupe la SEDm combinée à ce nouveau
modéliseur de scène dans cette \textit{data release}.

\end{document}

%%% Local Variables:
%%% mode: latex
%%% TeX-master: t
%%% End:
