\documentclass[../main/main.tex]{subfiles}
\begin{document}

%\dominitoc
%\faketableofcontents
\setcounter{chapter}{5}
\chapter{Réponse impulsionnelle de la SEDm et extraction de source ponctuelle}\label{ch:irf}
\minitoc
\vspace{2cm}
Le chapitre précédent était consacré à la modélisation hyperspectrale de
la galaxie hôte en utilisant localement le SED fitter \cigale\ sur des
images photométriques de PS1. Cette étape d'\hypergal\ nous fournit le
cube intrinsèque de la galaxie, composé de spaxels ayant chacun un
spectre qui lui est propre.

Les résolutions spectrales et spatiales ne sont cependant pas encore
adaptées à l'espace des observations dans lequel nous souhaitons projeter
le cube, à savoir celui de la SEDm. Nous devons pour cela considérer la
réponse impulsionnelle de notre instrument.

Par ailleurs, l'objectif d'\hypergal\ étant d'être un modéliseur de
scène, nous serons forcément amenés à modéliser la supernova. Cet objet
étant une source ponctuelle, elle est entièrement définie par le profil
de PSF, qui est la réponse impulsionnelle spatiale de la SEDm.

Dans ce chapitre nous commencerons par présenter la méthode de
détermination de la réponse impulsionnelle spectrale (LSF) de la SEDm,
et son application au cube intrinsèque. Puis nous introduirons un modèle
de profil radial pour la réponse impulsionnelle spatiale, que nous
entraînerons grâce à l'observation d'étoiles standards (sources
ponctuelles). Enfin nous procéderons à la validation de ce modèle de PSF par
une analyse de la calibration spectrophotométriques à partir de ces
étoiles standards.
\newpage

\section{Réponse impulsionnelle spectrale: LSF}
% \label{sec:xxx}

\subsection{Lampes à arc}
% \label{ssec:xxx}

Afin de caratériser la réponse impulsionnelle spectrale de la SEDm, nous
utilisons les lampes à arc que nous avons introduit dans le
chapitre~\ref{ssec:wavesol}.

Ces sources de lumière émettent un spectre avec d'intenses raies
d'émissions caractéristiques de l'élément présent dans la lampe.

Nous les utilisons initialement afin de déterminer la solution en longueur
d'onde de chaque trace spectrale sur le CCD, ce qui permet d'associer une longueur d'onde à une localisation spatiale
sur le détecteur du CCD.

Ce processus, détaillé dans \citet{pysedm} et le chapitre~\ref{ch:sedm}
de ce manuscrit, est effectué à l'aide de $3$ lampes à arc: au Xenon (Xe),
Mercure (Hg) et Cadmium (Cd). La combinaison de ces $3$ lampes permet de
couvrir tout le domaine spectral de la SEDm. La
Table~\ref{tab:linearclamp} détaille la position des raies pour chacune
des lampes.

\begin{table}[ht]
  \centerfloat
  \renewcommand{\arraystretch}{1.5}
    \begin{threeparttable}
        \caption{Raies d'émission lampes à arc}
        \label{tab:linearclamp}
        
        \begin{tabular}{lcccccc}
        \toprule
          \textbf{Lampe} &  Raie 1 &  Raie 2 & Raie 3 & Raie 4 & Raie 5 & Raie 6 \\
        \midrule
          \textbf{Hg} & $4047.7$  &  $4359.6$  &  $5462.3$  &  $5781.7^{\star}$  &  \ldots  & \ldots \\
          \textbf{Cd} & $4679.3$  & $4801.3$   & $5087.2$   & $6440.2$ &  \ldots  &  \ldots  \\
          \textbf{Xe} & $7644.1$  & $8250.1^{\star}$  & $8386.2^{\star}$  & $8821.8$  &  $9001.3^{\star}$  &  $9165.1$ \\
          \bottomrule
        \end{tabular}
        \begin{tablenotes}[flushleft]
        \item \textbf{Notes.} La notation $ ^{\star}$ correspond aux
          raies d'emissions qui résultent d'un mélange de plusieurs raies très
          rapprochées spectralement et non disernables par la SEDm.
        \end{tablenotes}
    \end{threeparttable}
  \end{table}

La LSF de la SEDm pouvant très bien être chromatique, nous allons
pouvoir tirer parti de la répartition de ces raies sur toute la plage
spectrale. La Figure~\ref{fig:arclamps} montre le spectre (moyenné sur
tout le MLA) des $3$ lampes à arc utilisées, en unité de flux par
longueur d'onde. La distribution des raies sur l'espace spectral permet
une excellente contrainte entre $4000$ et $6500$\AA\ grâce aux lampes Hg et
Cd. La lampe à Xenon permet, elle, de contraindre la solution en longueur
d'onde (et a fortiori la LSF dans cette étude) au delà de $7500$\AA.


\begin{figure}[h!]
  \centering
  \includegraphics[width=0.5\textwidth]{../figures/06_irf/arclamps.png}
  \caption[Spectres des lampes à arc utilisées pour la SEDm]{Spectres
    des lampes à arc utilisées pour la SEDm pour la nuit du 3 Juillet
    2020. De haut en bas, spectre de la lampe à mercure (Hg), à cadmium
    (Cd) et à Xenon (Xe). Ces spectres sont en unité de flux
    (pseudo-ADU) par longueur d'onde, et sont donc reconstruits à partir
  de la solution en longueur d'onde correspondante. Chaque spectre
  correspond au spectre moyen sur tout le MLA.}
  \label{fig:arclamps}
\end{figure}

\subsection{Détermination de la LSF}
% \label{ssec:xxx}

En toute rigueur, chaque spaxel possède sa propre réponse
impulsionnelle, et il faudrait déterminer la LSF pour chacun d'entre
eux. En pratique, nous faisons la supposition que la LSF moyenne sur
tout le MLA est suffisamment représentative de la réponse impulsionnelle
spectrale de la SEDm à l'échelle locale.

Pour prendre en compte une potentielle variation de la LSF au cours du
temps, nous utilisons les solutions en longueurs d'onde de $65$ nuits
étalées entre 2018 et 2022. Nous récupérons ainsi les positions et
écarts types modélisés pour chaque raie d'émission pour chaque spaxel de
chaque nuit. La solution en longueur d'onde nous permet également de
passer de l'espace des pixels du CCD à l'espace des longeurs d'onde.

Nous utilisons la médiane de la localisation ajustée de chaque raie
parmi tous les
spaxels, de même pour les écarts types, afin d'éviter les potentielles
valeurs aberrantes, notamment sur les bords du MLA (voir \citet{pysedm}). Les Figure~\ref{fig:lineloc} et
\ref{fig:linestd} montrent la distribution des localisations et écart
types médians des 65 nuits pour chaque raie d'émission.

On observe dans
la majorité des solutions en longueur d'onde un biais systématique par
rapport à la longueur d'onde de la raie de référence de l'ordre de
$3$\AA\ pour les lampes Cd et Hg . Les deux dernières raies de Xenon sont très peu contraintes, et
on peut apercevoir une dispersion de l'ordre de $20$\AA\ entre les
différentes nuits d'étude. 

La distribution des écarts types propres à chaque raie indique bien une
évolution chromatique, avec une résolution spectrale plus fine dans le
bleu que dans le rouge. La dispersion sur les nuits sélectionnées est de
l'ordre de quelques \AA.

\begin{figure}[h!]
  \centering
  \includegraphics[width=0.99\textwidth]{../figures/06_irf/lineloc.pdf}
  \caption[Distribution de la localisation des raies des lampes à
  arc]{Distribution de la localisation des raies des lampes à arc, en
    considérant la position médiane sur tous les spaxels du MLA pour la
    solution en longueur d'onde de 65 nuits entre 2018 et 2022. La
    localisation rouge verticale indique la position de la raie
    d'émission pour
    chaque lampe suivant les valeurs de la
    Table~\ref{tab:linearclamp}.}
  \label{fig:lineloc}
\end{figure}


\begin{figure}[h!]
  \centering
  \includegraphics[width=0.99\textwidth]{../figures/06_irf/linestd.pdf}
  \caption[Distribution de l'écart type $\sigma_{line}$ des raies des
  lampes à arc]{Distribution de l'écart type $\sigma_{line}$ des raies
    des lampes à arc, en
    considérant l'écart type médian sur tous les spaxels du MLA pour la
    solution en longueur d'onde de 65 nuits entre 2018 et 2022.}
  \label{fig:linestd}
\end{figure}

Sachant que les raies d'émission de la lampe à Xenon sont faiblement
contraintes, nous choisissons de modéliser la chromaticité de la LSF en
utilisant une combinaison linéaire de polynomes de Legendre de degré 2, afin d'éviter un effet
d'over-fitting aux extrémités.

Pour rappel, les polynômes de Legendre sont
constitués d'une suite de polynômes $p_{n}(x)$ de degré
$n$, et tous les polynômes de la suite sont orthogonaux deux à deux.

On peut les définir sous forme de somme tel que:

\begin{equation}
  \label{eq:legendre}
  P_{n}(x)=\frac{1}{2^{n}}\sum\limits^{n}_{k=0}\binom{n}{k}^{2}(x-1)^{n-k}(x+1)^{k}
\end{equation} 

Notre modèle de LSF chromatique est donc exprimé tel que:
\begin{equation}
  \label{eq:lsfmodel}
  \sigma(\lambda)=\sum\limits_{i}^{n=2}C_{i}P_{i}(\lambda)
\end{equation}
avec $P_{i}$ étant les polynomes de Legendre de degré $i$ et les $C_{i}$
les coefficients associés. Nous montrons dans la Figure~\ref{fig:lsf} la modélisation de la LSF
chromatique, avec les coefficients $C_{i}$ ajustés:

\begin{align*}
  C_{0}&=19.65\pm 0.32\\
  C_{1}&=26.0\pm 0.6\\
  C_{2}&=7.3\pm 0.6
\end{align*}

Pour des raisons de clareté visuelle, nous avons choisi de montrer sur
la Figure~\ref{fig:lsf} la distribution des écarts types sous forme de
violon, la dispersion de la position de la raie étant trop faible pour
être discernable sur la figure. Nous présentons la LSF
$\sigma_{\lambda}$ en unité de longueur d'onde mais également en unité
de pixel spectral pour la SEDm,
connaissant l'échantillonnage du domaine spectral.

\begin{figure}[h!]
  \centering
  \includegraphics[width=0.85\textwidth]{../figures/06_irf/LSF.pdf}
  \caption[Chromaticité de la LSF]{Chromaticité de la LSF. Nous montrons
  ici l'évolution de l'écart type $\sigma$ des différentes raies
  d'émission pour les lampes à arc Cd, Hg et Xe en fonction de la
  longueur d'onde. L'écart type $\sigma$ est présenté en unité de
  longueur d'onde ($\lambda$[\AA]) et en unité d'épaisseur de tranche
  dans les cubes 3D de la SEDm. Cette étude est réalisée à partir des
  solutions en longueurs d'onde de 65 nuits étalées entre 2018 et
  2022. La dispersion de la position des raies étant très faible et ne
  pouvant par conséquent être discernable sur la figure, nous présentons
la dispersion des $\sigma$ sous forme de violon. Le code couleur indique
la lampe à arc dont est issue la raie d'émission. Le modèle fitté
(polynomes de Legendre d'ordre 2) est présenté en courbe noir, et les
bandes grises indiquent l'erreur du modèle à 2 sigmas. }
  \label{fig:lsf}
\end{figure}

Sachant que la largeur à mi-hauteur (FWHM) pour une distribution
gaussienne est de $\text{FWHM}=2\sqrt{2\ln(2)}\sigma\approx2.355\sigma$, nous
pouvons également caractériser le pouvoir de résolution de la SEDm, que
nous illustrons dans la Figure~\ref{fig:resolutionsedm}.
Comme introduit dans \citet{SEDM18}, la résolution spectrale
$R=\frac{\lambda}{\Delta\lambda}$ est de l'ordre de $100$ sur tout le
domaine spectral. Néanmoins, nous pouvons apercevoir que cette
résolution spectrale décroît vers le rouge, avec
$R_{\lambda=4000\mathring{A}}\sim135$,
$R_{\lambda=6500\mathring{A}}\sim100$ et $R_{\lambda=8500\mathring{A}}\sim80$.\\

\begin{figure}[h!]
  \centering
  \includegraphics[width=0.9\textwidth]{../figures/06_irf/SEDmResolution_wmodel.pdf}
  \caption[Résolution de la SEDm]{Résolution de la SEDm, où
    $R=\frac{\lambda}{\Delta\lambda}$, avec $\lambda$ la longueur d'onde
  de la raie d'émission et $\Delta\lambda$ leur largeur à mi
  hauteur. Un violon correspond à la distribution en résolution à la
  position d'une raie d'émission (code couleur pour l'origine de la
  lampe) dans le même esprit que dans la Figure~\ref{fig:lsf}. Les lignes horizontales en pointillées indiquent les
  résolutions $R=135$, $100$ et $65$. La description de la SEDm
  \citep{SEDM18} indique une résolution $R\sim100$, ce qui est l'ordre
  de grandeur que nous retrouvons ici.}
  \label{fig:resolutionsedm}
\end{figure}


%\begin{figure}
%\centering
%\begin{subfigure}[b]{0.99\textwidth}
%  \includegraphics[width=1\linewidth]{../figures/06_irf/LSF.pdf}
%   \caption{Chomaticité de la réponse impulsionnelle spectrale (LSF) de
%     la SEDm}
%   \label{fig:Ng1} 
%\end{subfigure}

%\begin{subfigure}[b]{0.99\textwidth}
%   \includegraphics[width=1\linewidth]{../figures/06_irf/SEDmResolution_wmodel.pdf}
%   \caption{Résolution de la SEDm}
%   \label{fig:Ng2}
% \end{subfigure}
%\caption[blavla]{blablabla}
%\end{figure}

Nous fixons ainsi le modèle chromatique de LSF de la SEDm. Sachant que la résolution
des spectres obtenus avec \cigale\ est de l'ordre de $3$\AA\ sur
l'intervalle [$3200$-$9500$]\AA\ (correspondant à une résolution de
$R=\lambda/\d\lambda\approx2000$; \citet{BCO3}), nous choisissons
de convoluer directement les spectres du cube intrinsèque par la LSF de
la SEDm.

Cette convolution a donc la particularité d'être effectuée avec un
kernel gaussien ayant un écart type variable. Pour effecter
numériquement cette
transformation, nous procédons à un
étirement des spectres par l'inverse de la largeur du kernel à une
position donnée. Nous pouvons alors effectuer une convolution avec un
kernel fixe, puis ``reformer'' les spectres à leur échelle initiale.

La Figure~\ref{fig:lsfapplied} montre par exemple l'application du
modèle de LSF sur le spectre d'un spaxel du cube intrinsèque (le spectre
rouge de la Figure~\ref{fig:cigalesampling} après
ré-échantillonnage). Le lissage progressif et croissant avec la
longueur d'onde dû à la chromaticité de la LSF est
clairement visible.

\begin{figure}[h!]
  \centering
  \includegraphics[width=0.9\textwidth]{../figures/06_irf/lsfapplied.png}
  \caption[Application de la LSF]{Application de la LSF aux spectres du
    cube intrinsèque obtenu dans le Chapitre~\ref{sec:intcube}
    (Figure~\ref{fig:intcube_ZTF18accrorf}). Nous montrons ici un exemple
    de la convolution sur le spectre après ré-échantillonnage de la
    Figure~\ref{fig:cigalesampling} (ici en \textit{noir}). Le lissage progressif et croissant avec la
longueur d'onde dû à la chromaticité de la LSF est
clairement visible. Le résultat de la convolution par le kernel variable
est représenté par le spectre \textit{rouge}.}
  \label{fig:lsfapplied}
\end{figure}

Nous appliquons cette convolution pour tous les spaxels du cube
intrinsèque, ce qui nous permet d'avoir notre galaxie hôte dans l'espace
spectral de la SEDm.
\clearpage

\section{Réponse impulsionnelle spatiale: PSF}\label{sec:psf}

La section précédente fut consacrée à la caractérisation de la réponse
impulsionnelle spectrale de la SEDm. Nous présentons ici celle de la
réponse impulsionnelle spatiale, faite à partir d'observations d'étoiles standards (STDs) qui
sont des sources ponctuelles dans le champ de vue de l'IFS.

En associant l'imperfection du système optique et la présence des
turbulences atmosphériques \citep{Kolmogorov} dont l'hétérogénéité est dynamique, la structure d'une image
varie aléatoirement au cours du temps dans le champ de vue du
télescope. Une exposition de quelques secondes est habituellement
suffisante pour moyenner ces fluctuations et fixer l'image. Nous
pouvons alors relier  $O(\alpha,\lambda)$ le flux d'une source dans un direction $\alpha$ du
ciel à la longueur d'onde $\lambda$, et $\langle
I(\alpha,\lambda)\rangle$ la moyenne temporelle du flux observé
décrivant l'image obtenue dans le plan focal par la
transformation:

\begin{equation}
  \label{eq:transfertpsf}
  \langle I(\alpha,\lambda)\rangle = O(\alpha,\lambda) \otimes \langle
  S(\alpha, \lambda)\rangle
\end{equation}
avec $\langle S(\alpha, \lambda)\rangle$ l'image moyenne dans le plan focal d'une
source ponctuelle observée à l'infini, qui n'est autre que notre
fonction d'étalement de point (PSF). Sa transformée de Fourrier, notée
$\langle \widetilde{S}(\alpha, \lambda)\rangle$, est la fonction
de transfert du système optique dans son ensemble, qui n'est autre que
le produit entre la fonction de transfert du télescope achromatique
$T(f)$, et la fonction de transfert de l'atmosphère $B(\lambda, f)$.\\

La caractérisation de la PSF est donc cruciale pour une
modélisation de scène robuste avec \hypergal, étant donnée qu'une
supernova, qui fera partie des composantes de la scène, est elle
même une source ponctuelle à l'instar des STDs.

Cette section est divisée en trois parties. Dans un premier temps nous
présenterons le modèle de profil radial utilisé pour la PSF, puis nous
détaillerons l'entraînement de ce modèle destiné à le contraindre. Enfin,
nous aborderons l'aspect chromatique de cette réponse
impulsionnelle ainsi que les effets atmosphériques sur la localisation de la
source ponctuelle dans le MLA.


\subsection{Modèle de profil radial}\label{ssec:radialpsf}

Bien qu'il existe des modèles de PSF dérivés de la théorie des
perturbations atmosphériques \citep{Kolmogorov, Fried1966,Tokovinin},
leur capacité à décrire correctement les données n'est en général pas
suffisant. \citet{Butonthese} montre par exemple dans le cadre de
SNfactory que de tels modèles ne
permettent pas de bien représenter le ``coude'' séparant le coeur des ailes du profil radial.

Une simple gaussienne est parfois utilisée comme par \citet{King1971},
mais une telle représentation, quoiqu'efficace pour la représentation du
coeur d'une source ponctuelle ne permet pas d'ajuster les ailes du
profil radial.

Cette partie de la PSF peut cependant être modélisée par une
loi de puissance qui décroît moins vite que la gaussienne, comme
introduit par \citet{Moffat1969}. Des modèles de PSF basés sur cette
fonction homonyme (Moffat) ont par exemple était proposés par
\citet{Racine1996, Trujillo2001}.

Nous choisissons d'adopter la modélisation proposée dans la thèse de
\citet{Butonthese}, qui est également celle utilisée par \citet{pysedm}
pour la description de la PSF de la SEDm. Ce modèle empirique et
analytique a pour but d'introduire une composante pour chaque partie du
profil radial, à savoir une Gaussienne pour la description du coeur, et
une Moffat pour la description des ailes. Le modèle total est ainsi une
simple combinaison linéaire entre ces deux distributions:

\begin{equation}
  \label{eq:psfmodel}
  PSF(r) = N\left[\eta\times\exp\left(- \frac{r}{2\sigma^{2}}\right) +
    \left( 1+\left( \frac{r}{\alpha}\right)^{2}\right)^{-\beta} \right]
\end{equation}

Les paramètres $\eta$, $\sigma$, $\alpha$ et $\beta$ sont les paramètres
de forme du profil radial. Il faut cependant prendre également en compte
l'éventualité de défaut de focalisation et/ou de fortes variations de guidage du télescope, dont la conséquence sera
d'induire une ellipticité à notre source ponctuelle dans le plan focal,
et ne sera ainsi plus une image circulaire.

Le rayon $r$ de l'équation~\ref{eq:psfmodel} est ainsi un rayon
elliptique, tel que:

\begin{equation}
  \label{eq:ellipticity}
  r^{2}=r_{ell}^{2}=(x-x_{0})^{2}+\mathcal{A}(y-y_{0})^{2}+2\mathcal{B}(x-x_{0})\times(y-y_{0})
\end{equation}

Avec $x_{0}$ et $y_{0}$ les coordonnées du centre de la source
ponctuelle.

Les paramètres $\mathcal{A}$ et $\mathcal{B}$ décrivent simultanément le
rapport des deux axes $q$ et l'orientation de l'ellipticité $\phi$ tel
que:

\begin{equation}
  \label{eq:axesratioellipse}
  q=1-\frac{\left( \sqrt{(1-\mathcal{A})^{2}+4\mathcal{B}^{2}}-(1+\mathcal{A})\right)}{\left( -\sqrt{(1-\mathcal{A})^{2}+4\mathcal{B}^{2}}-(1+\mathcal{A})\right)}
\end{equation}\\

\begin{equation}
  \label{eq:angleellipse}
  \phi= \left\{
    \begin{array}{ll}
        \frac{1}{2}\cot^{-1}\left(\frac{1-\mathcal{A}}{2\mathcal{B}}\right) & \mbox{si } \mathcal{A}>1 \\
         \frac{\pi}{2}+\frac{1}{2}\cot^{-1}\left(\frac{1-\mathcal{A}}{2\mathcal{B}}\right) & \mbox{si } \mathcal{A}<1 
    \end{array}
\right.
\end{equation}


Ce formalisme décrit ainsi entièrement la source ponctuelle à
l'amplitude près. Il faut cependant également tenir compte du fond du
ciel, le \textit{background}. Cette composante ce rajoute donc au profil
de PSF. En temps normal, la composante du ciel est censée être uniforme,
et une constante devrait suffir à la modéliser. Dans notre cas, les
cubes extraits avec \pkg{pysedm} \citep{pysedm} présentent régulièrement
des artefacts indésirables notamment sur les bords du cube, et sont
d'autant plus intenses aux extrémités de l'espace spectral couvert par
la SEDm. Afin de palier à ces effets, nous introduisons un background
polynomial d'ordre 2, de sorte que :

\begin{equation}
  \label{eq:backgroundcurved}
  \text{Bkgd}(x,y) = (b_{xx}\times x^{2})+ (b_{yy}\times y^{2})+(b_{xy}\times xy)+(b_{x}\times x)+(b_{y}\times y) + b_{0}
\end{equation}

Avec $x$ et $y$ les coordonnées en spaxel de notre cube, et $b_{0}$ une
constante qui n'est autre que la composante qui décrit le fond de ciel.

Nous montrons dans la Figure~\ref{fig:radialprofile} un exemple de
profil radial fitté sur une meta-tranche à $6244$\AA\ pour la STD
25d4655, avec les différentes contributions de la fonction
d'étalement de point.
\begin{figure}[ht]
  \centering
  \includegraphics[width=0.99\textwidth]{../figures/06_irf/psfprofile.pdf}
  \caption[Exemple de profil radial d'un étoile standard]{Profil radial
    pour la méta-tranche à $6244$\AA\ de l'étoile standard 25d4655.}
  \label{fig:radialprofile}
\end{figure}

\subsection{Entrainement du modèle}\label{ssec:psftraining}

Notre modèle de PSF contient ainsi $4$ paramètres de forme ($\eta$,
$\alpha$, $\beta$, $\sigma$) et $2$ paramètres de focalisation décrivant
l'ellipticité et l'orientation. Ces $6$ paramètres sont cependant à
priori chromatiques, et le nombre de degré de liberté pour décrire une
simple source ponctuelle devient trop important.

Nous nous sommes ainsi penchés sur l'étude des corrélations entre ces
paramètres et leur chromaticité, afin de contraindre notre modèle de
PSF. Pour faire cela, nous avons utilisé environ $150$ cubes de
données d'étoiles standards, observées avec la SEDm en 2021.

Dans un premier temps, nous procédons à un ajustement avec la fonction
d'étalement de point entièrement libre, pour 9 méta-tranches indépendantes
entre $4500$ et $9000$ \AA. Nous avons choisi de ne pas considérer les
longueurs d'ondes au delà de ces extrémités à cause des artefacts trop intenses
générés lors de l'extraction des spectres du CCD, pouvant aller jusqu'à
masquer la source astronomique dans le champ de vue du MLA.

Nous avons ainsi commencé par chercher quels paramètres présentaient la
plus forte corrélation avec les autres. La
Figure~\ref{fig:stdcorrmatrix} met ainsi en évidence, d'une part, une très forte
corrélation entre $\alpha$ et $\beta$, mais également le fait que
$\alpha$ semble montrer le plus de corrélation avec les autres
paramètres de forme. Nous choisissons ainsi $\alpha$ comme paramètre
directeur de notre modèle de PSF.

\begin{figure}[ht]
  \begin{minipage}[c]{0.4\textwidth}
    \includegraphics[width=\textwidth]{../figures/06_irf/STD_correlation.pdf}
  \end{minipage}\hfill
  \begin{minipage}[c]{0.53\textwidth}
    \caption[Matrice de corrélation des paramètres de PSF.]{Matrice de
      corrélation des paramètres de PSF toutes méta-tranches confondues,
    avec tous les paramètres libres.}\label{fig:stdcorrmatrix}
  \end{minipage}
\end{figure}

L'idée est alors de fixer adéquatement les corrélations entre les
paramètres, puis de ré-entraîner le modèle de PSF avec ses nouvelles
contraintes. On vérifie alors à nouveau la présence ou non d'autres fortes
corrélations, et nous les fixons successivement.

Cet entrainement est également réalisé chromatiquement. En effet, même
si 2 paramètres sont fortement corrélés sur l'ensemble de l'interval
spectral étudié, nous ne
savons pas a priori si la forme de ces corrélations sont, ou non,
chromatiques. Nous analysons ainsi l'évolution de ces corrélations en
fonction de la longueur d'onde.

\subsubsection{Première contrainte: $\alpha$ vs $\beta$}

Le rayon ($\alpha$) et l'exposant ($\beta$) de la Moffat sont les deux
paramètres qui présentent la plus forte corrélation et de façon significative, nous commençons
donc par fixer celle ci. Nous présentons dans la
Figure~\ref{fig:alphabetachromcorr} les ajustements linéaires pour
chaque méta-tranche. Cet ajustement est effectué par minimisation de
$\chi^{2}$, en pondérant donc par les erreurs.


\begin{figure}[ht]
  \centering
  \includegraphics[width=0.8\textwidth]{../figures/06_irf/STD_alpha_beta_chromatic_corr.pdf}
  \caption[Chromaticité des corrélations entre $\alpha$ et $\beta$]{Chromaticité des corrélations entre $\alpha$ et $\beta$.}
  \label{fig:alphabetachromcorr}
\end{figure}

La chromaticité de ces ajustements est représenté dans la
Figure~\ref{fig:chromslope_zp_alphabeta}, où nous montrons l'évolution
du point zéro (ordonnée à l'origine) et de la pente en fonction de la
longueur d'onde de la meta-tranche considéré. On observe des effets
chromatiques de l'ordre de $12\%$ pour la pente, et de $6\%$ pour le
point zéro.
Nous avons choisi d'ignorer ces effets chromatiques, et de fixer
$\beta(\alpha)$ indépendamment de la longueur d'onde comme une
combinaison linéaire tel que

\begin{equation}
  \label{eq:betaalpha}
  \beta(\alpha) = \beta_{1}\times \alpha + \beta_{0}
\end{equation}
avec $\beta_{1}$ et $\beta_{0}$ fixés.

\begin{figure}[ht]
  \centering
  \includegraphics[width=0.7\textwidth]{../figures/06_irf/chromaticitybeta_alpha_corr.pdf}
  \caption[Chromaticité de la pente et du point zéro entre $\alpha$ et $\beta$]{Chromaticité de la pente et du point zéro entre $\alpha$ et $\beta$.}
  \label{fig:chromslope_zp_alphabeta}
\end{figure}

%\begin{figure}[ht]
%  \centering
%  \includegraphics[width=0.8\textwidth]{../figures/06_irf/STD_alpha_beta_corr.pdf}
%  \caption[]{}
%  \label{fig:corralphabeta_achrom}
%\end{figure}

\subsubsection{Seconde contrainte: $\alpha$ vs $\sigma$}


Après avoir fixé la corrélation entre $\alpha$ et $\beta$, on effectue
une nouvelle fois l'ajustement du modèle de PSF pour les mêmes étoiles
standards utilisées précédemment. Nous montrons dans la
Figure~\ref{fig:stdcorrmatrixbetafixed} la nouvelle matrice de
corrélation entre les paramètres de forme en négligeant la chromaticité,
le but étant juste d'avoir une estimation des paramètres présentant les
plus fortes corrélations.


\begin{figure}[ht]
  \begin{minipage}[c]{0.4\textwidth}
    \includegraphics[width=\textwidth]{../figures/06_irf/STD_correlation_matrix_betafixed.pdf}
  \end{minipage}\hfill
  \begin{minipage}[c]{0.53\textwidth}
    \caption[Matrice de corrélation des paramètres de PSF ($\beta(\alpha)$ fixé).]{Matrice de
      corrélation des paramètres de PSF toutes méta-tranches confondues,
    après fixation de $\beta(\alpha)$.}\label{fig:stdcorrmatrixbetafixed}
  \end{minipage}
\end{figure}

Nous nous intéressons donc à présent à la relation entre $\alpha$ (le
rayon de la Moffat) et $\sigma$, le rayon de la gaussienne.

De la même manière que précédemment, nous présentons dans la
Figure~\ref{fig:alphasigmachromcorr} les ajustements linéaires entre ces
deux paramètres pour chaque méta-tranche. Il est à noter que cette
corrélation est presque aussi significative que celle entre $\alpha$ et
$\beta$, ce qui montre à quel point ces trois paramètres sont corrélés
entre eux.


\begin{figure}[ht]
  \centering
  \includegraphics[width=0.8\textwidth]{../figures/06_irf/STD_alpha_sigma_chromatic_corr.pdf}
  \caption[Chromaticité des corrélations entre $\alpha$ et $\sigma$]{Chromaticité des corrélations entre $\alpha$ et $\sigma$.}
  \label{fig:alphasigmachromcorr}
\end{figure}

Tout comme précédement, la chromaticité de ces ajustements est représenté dans la
Figure~\ref{fig:chromslope_zp_alphasigma}, où nous montrons l'évolution
du point zéro et de la pente en fonction de la
longueur d'onde de la meta-tranche considérée. On observe cette fois ci des effets
chromatiques de l'ordre de seulement $3\%$ pour la pente, et de $8\%$ pour le
point zéro.
Nous avons à nouveau choisi d'ignorer ces effets chromatiques, et de fixer
$\sigma(\alpha)$ indépendamment de la longueur d'onde comme une
combinaison linéaire tel que

\begin{equation}
  \label{eq:betaalpha}
  \sigma(\alpha) = \sigma_{1}\times \alpha + \sigma_{0}
\end{equation}
avec $\sigma_{1}$ et $\sigma_{0}$ fixés.

\begin{figure}[ht]
  \centering
  \includegraphics[width=0.7\textwidth]{../figures/06_irf/chromaticitysigma_alpha_corr.pdf}
  \caption[Chromaticité de la pente et du point zéro entre $\alpha$ et $\sigma$]{Chromaticité de la pente et du point zéro entre $\alpha$ et $\sigma$.}
  \label{fig:chromslope_zp_alphasigma}
\end{figure}

%\begin{figure}[ht]
%  \centering
%  \includegraphics[width=0.8\textwidth]{../figures/06_irf/STD_correlation_betafixed.pdf}
%  \caption[]{}
%  \label{fig:corralphasigma_achrom}
%\end{figure}

\subsubsection{Poids relatif des distributions gaussienne/Moffat:
$\eta$}

Le dernier paramètre de forme libre de notre modèle de PSF est le poids
relatif entre la gaussienne et la Moffat, $\eta$. En refaisant le même
travail que précédemment, à savoir relancer l'ajustement du modèle de
PSF avec $\beta$ et $\sigma$ fixés en fonction de $\alpha$, nous nous
rendons compte une absence totale de corrélation entre $\alpha$ et
$\eta$, comme l'atteste la Figure~\ref{fig:alphaetachromcorr}.

Afin d'éviter un scénario similaire à celui rencontré par
\citet{Butonthese} avec le modèle de Kolmogorov, où le coude dans les données n'est pas
bien représenté par le modèle de PSF, nous choisissons de laisser $\eta$
libre dans notre fonction d'étalement de point.

\begin{figure}[ht]
  \centering
  \includegraphics[width=0.8\textwidth]{../figures/06_irf/STD_alpha_eta_chromatic_corr.pdf}
  \caption[Chromaticité des corrélations entre $\alpha$ et $\eta$]{Chromaticité des corrélations entre $\alpha$ et $\eta$.}
  \label{fig:alphaetachromcorr}
\end{figure}

\subsubsection{Profil radial contraint}

La Table~\ref{tab:betasigmapsf} présente les valeurs
obtenues pour la pointe et l'ordonnée à l'origine des ajustements
linéaires pour $\beta(\alpha)$ et $\sigma(\alpha)$. 

\begin{table}[ht]
  \centerfloat
  \renewcommand{\arraystretch}{1.5}
  \caption{Valeurs des paramètres des ajustements linéaires pour
    $\beta(\alpha)$ et $\sigma(\alpha)$.}
  \label{tab:betasigmapsf}
    \begin{threeparttable}
        \begin{tabular}{lcc}
        \toprule
          \textbf{Paramètre} & $\beta$  & $\sigma$ \\
        \midrule
          \textbf{Pente} & $0.22$  &  $0.39$  \\
          \textbf{Point Zéro} & $1.53$  & $0.42$    \\
          
          \bottomrule
        \end{tabular}
        %\begin{tablenotes}[flushleft]
        %\item \textbf{Notes.} 
        %\end{tablenotes}
    \end{threeparttable}
\end{table}

Avec les contraintes ainsi ajoutés, le profil radial de
l'équation~\ref{eq:psfmodel} s'écrit à présent:

\begin{equation}
  \label{eq:psfmodelconstraint}
  PSF(r; \alpha, \eta) = N\left[\eta\times\exp\left(- \frac{r}{2(\sigma_{0}+\sigma_{1}\times\alpha)^{2}}\right) +
    \left( 1+\left( \frac{r}{\alpha}\right)^{2}\right)^{-(\beta_{0}+\beta_{1}\times\alpha)} \right]
\end{equation}

Ce nouveau profil radial de la fonction d'étalement de point est ainsi
utilisable pour une extraction 2D d'une source ponctuelle, à une
longueur d'onde $\lambda$ donnée donc. Cependant le but est d'extraire
le spectre de la source ponctuelle. Intuitivement, nous pourrions
appliquer cette extraction 2D à toutes les tranches de notre cube de
donnée. Mais ce processus, en plus de demander énormément de ressources
numériques, supposerait d'une part que les tranches ne sont pas corrélées
entre elles, et d'autre part que le ratio signal sur bruit serait
suffisant à l'échelle d'un pixel spectral. Or cela est généralement faux
dans les deux cas.

Nous présentons ainsi une méthode d'extraction adéquate dans la section suivante. 


\section{Extraction de la source ponctuelle}

\subsection{Méthode d'extraction}\label{ssec:methodextraction}

Puisqu'il n'est pas question d'extraire le flux d'une source ponctuelle
à chaque tranche du cube de donnée, l'idée est plutôt d'effectuer un
ajustement de la PSF sur $N$ méta-tranches du cube de
données, où le signal sur bruit est suffisamment élevé, et ainsi récupérer
un jeu de $N\times2D$ paramètres décrivant la PSF de la source et sa position 
dans le MLA.

En modélisant adéquatement la chromaticité de ces paramètres, nous
fixons alors tous les paramètres d'ajustement sur l'ensemble du domaine
spectral. In fine, nous extrayons le spectre de la source
ponctuelle en ne laissant libre que les paramètres d'amplitude et de
background pour chaque tranche du cube de donnée.

Nous devons ainsi modéliser la chromaticité de $5$ paramètres: les
paramètres de forme $\alpha$ et $\eta$, les paramètres d'ellipticité et
d'orientation $\mathcal{A}$ et $\mathcal{B}$, et l'évolution de la
position ($x_{0}$,$y_{0}$) de la source ponctuelle dans le MLA, causée
par la réfraction atmosphérique.

Commençons par aborder cet effet, présent indépendemment de l'étalement
du point.


\subsection{R\'efraction atmosph\'erique
  diff\'erentielle}\label{ssec:adr}


L'atmosphère ayant un indice de réfraction différent de celui du vide
spatial, la lumière d'une source astronomique nous parvenant sur Terre
est ainsi réfractée (3$\ieme$ loi de Snell/Descartes). Cet indice étant dépendant de la longueur d'onde, la
réfraction induite par le passage de la lumière dans l'atmosphère va
elle aussi être chromatique: chaque longueur d'onde est ainsi déviée
avec un angle de réfraction différent, à la manière d'un prisme. C'est
cet effet que l'on appelle réfraction atmosphérique différentielle
(\textit{Atmospheric Differential Refraction}; ADR).

On observe ainsi en spectroscopie un déplacement du centroïde des
sources astronomiques du champ de vue le long des tranches spectrales.

Le phénomène d'ADR dépend des conditions d'observations, en prenant en
compte d'une part la position de l'objet dans le ciel par rapport au
détecteur (contribution achromatique), et d'autre part
l'indice de réfraction de l'atmosphère (contribution chromatique).

Cet indice de réfraction varie avec la longueur d'onde de la lumière
incidente, mais également avec différents paramètres de l'atmosphère. Il
est donc nécessaire de bien connaître les différentes dépendances entre
l'indice de réfraction et les conditions atmosphériques lors de
l'observation pour modéliser correctement les effets de l'ADR.

Nous utilisons pour cela les équations de \citet{Edlen1966} modifées
par \citet{Birch1993, Birch1994} et référencées par \citet{Stone2001}\footnote{\url{https://emtoolbox.nist.gov/Wavelength/Documentation.asp}}.

Nous présentons ci dessous les équations permettant de remonter à
l'indice de réfraction. On notera $t$ pour la température (Celsius),
$p$, $p_{v}$ et $p_{sv}$ pour les pressions (Pascal) atmosphérique,
partielle et saturante de
vapeur d'eau, et enfin RH l'humidité relative.

On commence par déterminer $p_{v}$.
En définissant les constantes et quantités:

\begin{center}
  \renewcommand{\arraystretch}{1.5}
  \begin{tabular}{ll}
   
$A_1= -13.928169$ & $A_2 = 34.7078238$ \\
$T= t + 273.15$ & $t_h = \frac{T}{273.16}$ \\
  \multicolumn{2}{c}{$Y = A_1 \times(1 - t_h^{-1.5}) + A_2 \times (1 - t_h^{-1.25})$}
\end{tabular}\\
\end{center}

On définit la pression saturante de vapeur d'eau par:
\begin{equation}
  \label{eq:pvs}
    p_{sv}(t) = 611.657 \times e^{Y}
\end{equation}

On déduit la pression partielle de vapeur d'eau $p_v$ à partir de
l'humidité relative $RH$ (en pourcentage) par:
\begin{equation}
 \label{eq:pv}
    p_v(RH,t) = \frac{RH}{100}\times p_{vs}(t)
\end{equation}

Passons maintenant à la détermination de l'indice de réfraction $n(\lambda,RH,t,p)$
On définit dans un premier temps les $7$ constantes ci dessous:

\begin{center}
  \renewcommand{\arraystretch}{1.5}
\begin{tabular}{lll}
$A= 8342.54$ & $B = 2406147$ &\\
$C= 15998$ & $D = 96095.43$& $G = 0.003661$\\
$E = 0.601 $& $F = 0.00972$ &  \\
\end{tabular}
\end{center}

On défini alors les quantités intermédiaires suivantes, avec la longueur
d'onde $\lambda$ en \textmu m:
\begin{align*}
    S =&  \lambda^{-2}\\
    n_s =& 10^{-8}\left(\frac{A + B}{130 - S} + \frac{C}{38.9 - S}\right)\\
    X =& \frac{\left(1 + 10^{-8} \times (E - F\times t) \times
         p\right)}{(1 + G \times t)}\\
    n(\lambda,t,p) =& 1 + p \times n_s \times \frac{X}{D}
\end{align*}

Avec $n(\lambda,t,p)$ l'indice de réfraction en négligeant la contribution de
l'humidité relative. On exprime alors l'indice de réfraction avec toutes
les dépendances par:
\begin{equation}
  \label{eq:refractindex}
    n(\lambda,RH,t,p) = n(\lambda,t,p) - 10^{-10}\times \left(\frac{292.75}{t + 273.15}\right) \times \left(3.7345 - 0.0401 \times S\right)\times p_v(RH,p,t)
\end{equation}

Nous pouvons maintenant déterminer la déviation du centroïde de nos
objets dans le MLA à partir des indices de réfraction à une longueur
d'onde donnée, et une longueur d'onde de référence.

En notant ($x_{0}$, $y_{0}$) les coordonnées spatiales à la longueur
d'onde de référence $\lambda_{ref}$, les nouvelles coordonnées ($x_{\lambda}$, $y_{\lambda}$) à
la longueur d'onde observée dues aux effets de l'ADR sont déterminées
par la transformation:

$$
  \left\{
    \begin{array}{ll}
    x(\lambda)&= x_{ref} - \frac{1}{2}\left( \frac{1}{n^{2}(\lambda)} - \frac{1}{n^{2}(\lambda_{ref})}\right)\times \tan(d_{z})\sin(\theta) \\
    y(\lambda)&= y_{ref} - \frac{1}{2}\left( \frac{1}{n^{2}(\lambda)} - \frac{1}{n^{2}(\lambda_{ref})}\right)\times \tan(d_{z})\cos(\theta)
    \end{array}
   \right.
$$

Avec $\theta$ l'angle parallactique et $d_{z} = \arccos{\left(\chi^{-1}\right)}$ la
distance zénithale dans l'approximation d'une atmosphère plan-parallèle,
et $\chi$ la masse d'air le long de la ligne de visée ($\chi=1$
correspondant à un objet parfaitement au zénith).


\subsection{Ajustement chromatique}\label{ssec:chrom}

L'ADR et notre modèle de profil radial ayant été rigoureusement définis,
nous pouvons à présent procéder à l'ajustement chromatique de notre
source ponctuelle afin d'effectuer l'extraction 3D de son spectre.

Dans un premier temps, nous procédons à l'ajustement de la PSF en
incluant le ciel pour $9$ méta-tranches comprises entre $4500$ et
$9000$\AA. L'épaisseur ainsi obtenue ($\sim500$\AA) permet d'avoir un
signal sur bruit suffisamment élevé sans que l'ADR ne produise d'impact
significatif.

Pour chacune de ces méta-tranches, les paramètres d'amplitudes de la PSF
et les $6$ coefficients du background polynomial (équation~\ref{eq:backgroundcurved}) sont des paramètres de nuisance.

La Figure~\ref{fig:allmetastd} illustre l'ajustement de la fonction
d'étalement de point (profil radial + coutours) de chacune des méta-tranches pour l'étoile
standard 25d4655.

\begin{figure}[ht]
  \centering
  \includegraphics[width=0.95\textwidth]{../figures/06_irf/STD_profile_allmeta.pdf}
  \caption[Profil radial et coutours des $9$ metaslices de la STD
  25d4655]{Profil radial (\textit{à gauche}) et coutours d'intensité
    (\textit{à droite}) des $9$ metaslices de la STD
    25d4655. Les traits pleins du contour représentent les données et
    les pointillées l'ajustement du modèle.}
  \label{fig:allmetastd}
\end{figure}

La déviation chromatique du centroïde de la source ponctuelle due à
l'ADR est présentée dans la Figure~\ref{fig:adr_std}. L'estimation des
positions de références ($x_{ref}$, $y_{ref}$) associées aux 
paramètres de masse d'air et d'angle parallactique permet ainsi de
dériver la position de l'étoile dans le MLA à n'importe quelle longueur d'onde.

\begin{figure}
  \centering
  \includegraphics[width=0.6\textwidth]{../figures/06_irf/adr_std.pdf}
  \caption[Modélisation de la réfraction atmosphérique
  différentielle]{Modélisation de la réfraction atmosphérique
    différentielle pour l'étoile standard 25d4655. Nous montrons ici
    l'ajustement du centroïde ($x$,$y$) de la PSF par le modèle
    d'ADR. Les deux graphes du haut représentent l'ajustement des deux
    coordonnées en fonction de la longueur d'onde. Le graphe du bas
    illustre l'effet de la réfraction atmosphérique.}
  \label{fig:adr_std}
\end{figure}

Il ne nous manque ainsi plus qu'à fixer la chromaticité de la fonction
d'étalement de point.
Nous présentons dans la Figure~\ref{fig:chromaticity_psf} l'évolution
chromatique des paramètres de forme de la PSF. Nous ajustons les
paramètres d'ellipticité et d'orientation $\mathcal{A}$ et $\mathcal{B}$
par une constante, leur évolution étant relativement faible avec la longueur d'onde.

Le poids entre la gaussienne et la Moffat $\eta$ est également ajusté
par une constante. Bien que nous pouvons apercevoir des variations de
l'ordre de $5$ à $10\%$ autour de la moyennne pondérée, nous n'observons pas
de tendance chromatique dans son évolution.

En ce qui concerne le paramètre de forme principal $\alpha$, nous
utilisons pour l'ajustement une loi de puissance de la forme:

\begin{equation}
  \label{eq:alphachrom}
  \alpha(\lambda)=\alpha_{ref}\left(\frac{\lambda}{\lambda_{ref}}\right)^{\rho}
\end{equation}
avec $\alpha_{ref}$ et $\rho$ les paramètres d'ajustement de la chromaticité.

\begin{figure}
  \centering
  \includegraphics[width=0.7\textwidth]{../figures/06_irf/chromaticity_psf.pdf}
  \caption[Chromaticité des paramètres de forme de la PSF]{Ajustement de
    la chromaticité
    des paramètres de forme de la PSF pour l'étoile standard
    25d4655. \emph{À gauche} l'ajustement du paramètre $\alpha$ avec une
  loi de puissance. \emph{À droite} de haut en bas, l'ajustement par une
constante de $\eta$ (poids entre la gaussienne et la Moffat) et des
paramètres d'ellipticité ($\mathcal{A}$) et d'orientation ($\mathcal{B}$). }
  \label{fig:chromaticity_psf}
\end{figure}

La Figure~\ref{fig:stdspectrumadu} montre finalement le spectre extrait de
l'étoile standard 25d4655 en pseudo-ADU, qui est utilisé pour la calibration
en flux. 

\begin{figure}
  \centering
  \includegraphics[width=0.7\textwidth]{../figures/06_irf/stdspectra_adu.pdf}
  \caption[Spectre extrait de l'étoile standard 25d4655 en
  pseudo-ADU.]{Spectre extrait de l'étoile standard 25d4655 en
    pseudo-ADU avec \hypergal. Les bandes correspondent aux zones d'absorption
    tellurique d'O$_{2}$ et d'H$_{2}$O \citep{Buton2013}.}
  \label{fig:stdspectrumadu}
\end{figure}


\clearpage
\section{Calibration en flux}\label{sec:validationpsf}

\subsection{Méthode}\label{ssec:photocalibstd}

Afin d'avoir une estimation de la précision de la calibration en flux
avec notre modèle de PSF, nous utilisons la
méthode de calibration implémentée dans \pkg{pysedm} \citep{pysedm} décrite dans le
chapitre~\ref{ssec:calibpysedm}.

On rappelle que, pour la SEDm, le formalisme utilisé pour décrire le
spectre observé $S(\lambda,t,z)$ d'une source astronomique est tel que:

\begin{equation*} 
  S(\lambda,t,z)=S^{\star}(\lambda,t)\times\mathcal{C}_{atm}(\lambda,t,z)\times\mathcal{C}_{inst}(\lambda,t)\times\mathcal{T}(\lambda,t,z)
\end{equation*}
avec $S^{\star}(\lambda,t)$ le spectre instrinsèque de la source en
unités physiques ($erg/cm^2/s/$\AA), $\mathcal{C}_{atm}(\lambda,t,z)$
l'extinction atmosphérique, $\mathcal{C}_{inst}(\lambda,t)$ la réponse
instrumentale et $\mathcal{T}(\lambda,t,z)$ l'absorption tellurique. L'extinction
atmosphérique utilisée pour le Mont Palomar est celle de
\citep{Hayes1975atm}, et est appliquée lors de la création des cubes
SEDm. Le spectre extrait en pseudo-ADU de la
Figure~\ref{fig:stdspectrumadu} est donc déjà corrigé de cette composante.

L'absorption tellurique est re-paramétrisé de sorte que $\mathcal{T} =
1-\mathcal{E}_{tell}$, avec $\mathcal{E}_{tell}$ le débit tellurique, ce qui nous donne finalement:
\begin{equation*} 
  S_{ADU}(\lambda,t,z) = \frac{S(\lambda,t,z)}{\mathcal{C}_{atm}(\lambda,t,z)}=S^{\star}(\lambda,t)\times\mathcal{C}_{inst}(\lambda,t)\times\left(1-\mathcal{E}_{tell}(\lambda,t,z)\right)
\end{equation*}

On utilise comme référence le spectre spectrophotométrique
correspondant à l'étoile standard observée, obtenu dans les archives
calspec\footnote{\url{https://archive.stsci.edu/hlsps/reference-atlases/cdbs/current_calspec/}}.

Les spectres
telluriques utilisées sont ceux du Kitt Peak National
Observatory\footnote{\url{http://www.noao.edu/kpno/}} \citep{Hinkle2003}, scindés en deux
catégories de longueur d'onde: l'$\text{O}_{2}$ et
l'$\text{H}_{2}\text{O}$. La Figure~\ref{fig:telluriclines} montre les
raies d'absorption telluriques utilisées.

\begin{figure}[ht]
  \centering
  \includegraphics[width=0.85\textwidth]{../figures/06_irf/telluricspec.pdf}
  \caption[Raies d'absorption telluriques]{Raies d'absorption
    telluriques du Kitt Peak National Observatory \citep{Hinkle2003}, avec les composantes d'$\text{O}_{2}$ et
    d'$\text{H}_{2}\text{O}$. La projection dans l'espace spectral de la
  SEDm est montrée en rouge.}
  \label{fig:telluriclines}
\end{figure}

La dépendance en masse d'air de l'absorption tellurique est exprimée suivant:

\begin{equation*}
  \label{eq:telluricpysedm2}
  \mathcal{T}(z) =
  \mathcal{T}_{\text{O}_{2}}\times(c_{\text{O}_{2}}+z^{\rho_{\text{O}_{2}}})
  + \mathcal{T}_{\text{H}_{2}\text{O}}\times(c_{\text{H}_{2}\text{O}}+z^{\rho_{\text{H}_{2}\text{O}}})
\end{equation*}
où les amplitudes relatives $c_{i}$ et les coefficients
$\rho_{i}$ sont des paramètres libres. La réponse instrumentale
$\mathcal{C}_{inst}$ est quant à elle modélisée par un polynome de Legendre
d'ordre $20$.


Les composantes de réponse instrumentale $\mathcal{C}_{inst}$ et
d'absorption telluriques $\mathcal{T}$ sont alors simultanément
ajustées en minimisant la quantité ~$(S_{ADU}/S_{ref}) - \mathcal{C}_{inst}(1-
\mathcal{E}_{tell})$, où la quantité $\mathcal{C}_{inst}(1-
\mathcal{E}_{tell})$ est appelée la courbe de
sensibilité.

Ce procédé d'ajustement en distingant les composantes telluriques de la
réponse instrumentale est nécessaire de par la dépendance en masse
d'air de l'absorption tellurique. Lorsque l'on applique la
calibration en flux ainsi obtenue sur une observation scientifique, nous
pouvons ajuster la contribution tellurique en considérant la masse d'air
présente lors de l'observation.

La Figure~\ref{fig:calibmodel} illustre ainsi l'ajustement de ces deux
contributions pour l'étoile standard 25d4655, après ré-échantillonnage
des raies telluriques et du spectre spectrophotométrique Calspec dans
l'espace spectral de la SEDm.

Dans ce cas particulier, nous pouvons
observer une légère déviation en longueur d'onde lors de l'ajustement
des absorptions telluriques. Ce phénomène est assez rare, et résulte
d'un mauvais alignement en longueur d'onde
à partir des pixels du CCD qui a été mal/non
corrigé lors de l'extraction du cube 3D (voir
Chapitre~\ref{ssec:3dcubecons}, étape (d)). \citet{pysedm} fait
également part d'un biais systématique de $\sim3$\AA\ dans la
calibration en longueur d'onde en cours d'investigation. L'effet est d'autant plus
exacerbé par l'intensité de la raie d'O$_{2}$ à $\sim7600$\AA.

\begin{figure}[ht]
  \centering
  \includegraphics[width=0.85\textwidth]{../figures/06_irf/calibmodel.pdf}
  \caption[Procédure d'ajustement de la calibration en flux]{Procédure
    d'ajustement de la calibration en flux pour l'étoile standard
    25d4655. \emph{En haut} est illustré le spectre extrait de l'étoile
    standard en bleu, et le spectre de référence en orange. \emph{En
      bas} nous montrons le résultat de l'ajustement du modèle composé
    de la réponse instrumentale (en rouge) et de l'absorption tellurique
  (en gris). Les données sur lesquelles l'ajustement est effectuée, le
  ratio $\frac{S_{ADU}}{S_{ref}}$, est en bleu.}
  \label{fig:calibmodel}
\end{figure}

\subsection{Précision de la calibration}\label{ssec:resultscalib}

L'ajustement du modèle de sensibilité inversé présenté dans la section
précédente permet ainsi de calibrer en flux les observations
scientifiques. Afin d'estimer la précision de cette calibration, nous
observons à nouveau une étoile standard (dans la même nuit ou la
suivante), nous procédons à l'extraction de son spectre en pseudo-ADU,
puis nous appliquons la calibration obtenue avec la précédente étoile.
Cela nous permet alors de comparer cette étoile calibrée à partir d'une
observation antérieur, avec son spectre de référence
spectrophotométrique.

Nous avons exploité tout ce chapitre l'étoile standard 25d4655, observée
le 24/08/2021 à 10h 04m 49s. La même étoile a été ré-observée la même
nuit une demi-heure plus tard, à 10h 38m 16s.

Nous pouvons ainsi extraire le spectre en pseudo-ADU de cette seconde
observation tel que détaillé dans la première section de ce chapitre,
lui appliquer la calibration en flux déterminée à partir de la première
observation, et ainsi vérifier la précision obtenue en comparant le
spectre calibré avec celui de référence Calspec.
Nous montrons ce résultat dans la Figure~\ref{fig:fluxcalstd}.
Le RMS sur tout l'espace spectral est de $1.87\%$, avec une distribution
du ratio entre les deux spectres n'excédant pas les $5\%$. Cette
déviation apparaît notamment à la localisation des raies telluriques les
plus intenses (O$_{2}$ à $\sim7600$\AA), due à une extraction du cube de
la première observation de l'image CCD mal corrigée.

\begin{figure}[ht]
  \centering
  \includegraphics[width=0.85\textwidth]{../figures/06_irf/fluxcalstd.pdf}
  \caption[Précision de la calibration en flux pour une étoile
  standard.]{Précision de la calibration en flux pour l'étoile standard
    25d4655 (ID:10-38-16) à partir de l'observation de la même étoile standard
    observée $\sim33$min plus tôt.\emph{En haut} sont présentés le
    spectre extrait calibré (bleu) et celui de référence (orange) en
    unités physiques. Les bandes bleues indiquent les zones
    d'absorptions telluriques les plus intenses, dues à l'O$_{2}$. En
    bas est représenté le ratio entre les flux. Le RMS spectral pour
    cette calibration est de $1.87\%$.}
  \label{fig:fluxcalstd}
\end{figure}


Dans ce cas de figure, les deux observations ayant eu lieu sur la même
étoile à seulement une demi-heure d'intervalle nous avons supposé que
l'extinction atmosphérique était constante avec le temps (faible délai
entre les observations) et uniforme (faible déviation de la ligne de
visée dans le ciel).

Nous avons ainsi extrait le spectre de plus de $2000$ étoiles standard
observées avec la SEDm entre Juin 2018 et Février 2022, et procédé
à la méthode de calibration expliquée précédemment pour chaque paire d'étoile observée
successivement dans le temps.
Afin de corriger une éventuelle variation d'extinction atmosphérique
entre deux observations, nous avons incorporé un terme gris
(achromatique) normalisant ainsi le spectre extrait et celui de référence.

La Figure~\ref{fig:allratio_std} présente ainsi le ratio (normalisé par
un terme gris) entre les spectres extraits et calibrés en flux avec \hypergal\ de 2202 étoiles
standards et leur spectre de référence Calspec ré-échantillonné dans
l'espace spectral de la SEDm. On y présente le ratio moyen, dont la
distribution oscille autour de $1$-$2\%$. En considérant également les
enveloppes à $1\sigma$ et $2\sigma$, ces résultats nous indiquent une
calibration de couleur de l'ordre de quelques pourcents, majoritairement
en dessous des $5\%$. On remarque également que la présence des raies
telluriques à $\sim7600$\AA\ est fortement présente dès que l'on
considére l'enveloppe à $1\sigma$, ce qui conforte l'hypothèse d'une
erreur systématique dans les solutions en longueur d'onde.

Nous pouvons également remarquer la dégradation importante aux
extrémités du domaine spectral de la SEDm, notamment en deça de
$4500$\AA, et au dessus de $8500$\AA.

\begin{figure}[ht]
  \centering
  \includegraphics[width=0.99\textwidth]{../figures/06_irf/fluxcalstd_all_ratio.pdf}
  \caption[Moyenne du ratio entre le flux de 2202
  étoiles standards et leur spectre de
  référence.]{Moyenne du ratio entre le flux de 2202
    étoiles standards calibrées avec \hypergal et leur spectre de
    référence Calspec. Les enveloppes bleues représentent les déviations
    standard à $1\sigma$ et $2\sigma$. Les traits horizontaux représentent
    les limites $\pm5\%$ et l'unité.}
  \label{fig:allratio_std}
\end{figure}

La Figure~\ref{fig:rms_allstd} présente la distribution du RMS spectral
des $2202$ calibrations 
pour différents intervalles de longueur d'onde. Les caractéristiques de
ces distributions sont présentées dans la Table~\ref{tab:rms_allstd_detail}. En
considérant le domaine spectral de la SEDm dans son ensemble, la
calibration en flux obtenue avec \hypergal\ est de l'ordre de $3.3\%$. En
limitant le domaine spectral à $\lambda\in$[$4500$-$8500$]\AA\, nous
atteignons une précision de l'ordre de $2\%$ avec une faible déviation
standard de $0.7\%$. Les résultats sont similaires avec le domaine
spectral $\lambda\in$[$5000$-$8000$]\AA. Dans les deux derniers
intervalles spectraux considérés, le RMS spectral n'excède pas les $4\%$. 

\begin{figure}[ht]
  \centering
  \includegraphics[width=0.99\textwidth]{../figures/06_irf/rmsallstd_fluxcalibration.pdf}
  \caption[Distributions du RMS spectral des
  calibrations en flux pour différents intervalles spectraux]{Distributions du RMS spectral des
  calibrations en flux de 2202 étoiles standards pour différents
  intervalles spectraux. \emph{En haut} sont représentés les
  histogrammes des distributions et \emph{en bas} le box plot
  correspondant. La distribution \emph{bleu} correspond au domaine
  spectral entier de la SEDm, en \emph{orange} l'intervalle
  $\lambda\in$[$4500$-$8500$]\AA, et en \emph{vert} l'intervalle
  $\lambda\in$[$5000$-$8000$]\AA. Les box de la figure du bas montre
  pour chaque distribution le minimum, le $1^{er}$ quartile ($25\%$ des
  valeurs), la médiane, le $3^{ème}$ quartile ($75\%$ des
  valeurs) et le maximum.}
  \label{fig:rms_allstd}
\end{figure}

\begin{table}[ht]
  \centerfloat
  \renewcommand{\arraystretch}{1.4}
  \caption[Description des distributions du RMS spectral des
  calibrations en flux.]{Description des distributions du RMS spectral de la
    Figure~\ref{fig:rms_allstd} pour chaque intervalle de longueur d'onde.}
  \label{tab:rms_allstd_detail}
  \begin{threeparttable}
    \begin{tabular}{cccc}
      \toprule
      \textbf{RMS spectral (\%)} & $\lambda \in [3700$-$9300]$\AA & $\lambda \in [4500$-$8500]$\AA  &$\lambda \in [5000$-$8000]$\AA \\
      \midrule
      Moyenne  &                                       3.35 &                                       2.17 &                                       2.04 \\
      Déviation standard   &                                       1.34 &                                       0.70 &                                       0.59 \\
      Minimum   &                                       1.19 &                                       1.05 &                                       0.97 \\
      $1^{er}$ quartile (25\%)   &                                       2.25 &                                       1.60 &                                       1.55 \\
      Médiane (50\%)   &                                       3.07 &                                       2.03 &                                       1.97 \\
      $3^{rd}$ quartile (75\%)   &                                       4.30 &                                       2.63 &                                       2.45 \\
      Maximum   &                                       6.62 &                                       4.03 &                                       3.56 \\
      \bottomrule
    \end{tabular}

    % \begin{tablenotes}[flushleft]
    % \item \textbf{Notes.} 
    % \end{tablenotes}
  \end{threeparttable}
\end{table}

  

%\bibliographystyle{../main/aa_url2}
%\bibliography{99_references}

\end{document}

%%% Local Variables:
%%% mode: latex
%%% TeX-master: t
%%% End:
