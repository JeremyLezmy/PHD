\documentclass[../main/main.tex]{subfiles}
\begin{document}


\chapter*{Résumé}
\label{chap:resume}
\vspace{2cm}
Il y a près de deux décennies, les distances mesurées à partir des
supernovae de type Ia (SNeIa) ont été utilisées pour découvrir
l’accélération de l’expansion de l’Univers. Alors que ce phénomène est
désormais communément admis, la compréhension de ses causes demeure
inconnue et nécessite des mesures précises et non biaisées.
Les supernovae (SNe) sont réparties en plusieurs catégories discernables
par leurs propriétés spectrales, et de tous les types de SN, seules les
SNeIa sont utilisées comme sonde cosmologique de par leurs propriétés de
chandelles standardisables.
Avec différents types de SNe possédant des luminosités intrinsèques
différentes, tout objet mal classifié peut induire des biais dans la
dérivation des paramètres cosmologiques.

Actuellement la classification spectrale n'est possible que lorsque la SN est suffisamment
isolée de sa galaxie hôte, mais plus l'évènement survient proche du cœur
de la galaxie plus la contamination spectrale s'intensifie et cette analyse
devient difficile voire impossible.

Cette thèse de doctorat a été effectuée à l’Institut de Physique des 2
Infinis de Lyon (IP2I Lyon) dans le cadre du relevé cosmologique grand champ \textit{Zwicky
Transient Facility} (ZTF), avec l'utilisation d’un spectrographe 3D
basse résolution: la \textit{Spectral Energy Distribution machine}
(SEDm). L’objectif de ce travail de recherche est de répondre à la
problématique de la classification des SNe dans le cas de
contamination spectrale importante par la galaxie hôte. 

Je présente dans ce manuscrit une nouvelle
méthode de décontamination sous la forme d'un outil de modélisation de
scène, \hypergal, permettant d'extraire le spectre de supernovae jusqu'alors
inexploitables. Le cœur de ce pipeline repose sur l’utilisation de
données photométriques de la galaxie hôte, prises en amont de
l’explosion de la SNIa. En utilisant les connaissances de la physique
des galaxies, nous modélisons à l’échelle locale les propriétés
spectrales de l'hôte afin de créer un cube 3D modèle de la galaxie
isolée. En convoluant ce modèle par la réponse impulsionnelle de la
SEDm, nous effectuons une projection de cette modélisation
hyperspectrale dans l'espace des observations. Modélisant par la suite
la SN par une source ponctuelle chromatique, et ajustant simultanément la
galaxie et la supernova aux observations, nous avons créé un pipeline
modéliseur de scène pouvant extraire le spectre de la SN dans un
environnement hautement contaminé. Cet outil est par la suite validé sur un
échantillon simulé de SNe tirées de données observées.

J’expose à la fin de ce manuscrit les résultats scientifiques
préliminaires de la seconde \textit{data release} (DR2) du groupe \textit{Type Ia
Supernovae $\&$ Cosmology} de ZTF, composée de $\sim3000$ SNeIa et dont la vaste majorité ont
été classifiées avec cette nouvelle méthode d’extraction.


\end{document}

%%% Local Variables:
%%% mode: latex
%%% TeX-master: t
%%% End:
