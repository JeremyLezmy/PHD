\documentclass[../main/main.tex]{subfiles}
\begin{document}
\dominitoc
\faketableofcontents
\setcounter{chapter}{6}
\chapter{Modélisation de scène et extraction de sources}\label{ch:res}
\minitoc
\vspace{2cm}
Ce chapitre est consacré à la description de la dernière étape du pipeline \hypergal, la
modélisation de scène. Les chapitres précédents ont permis dans un
premier temps la
construction du cube intrinsèque de la galaxie hôte. Puis nous avons
procédé à sa projection dans
l'espace spectral de la SEDm à partir de la réponse impulsionnelle
spectrale de l'instrument. Enfin, nous avons également construit un modèle
de PSF robuste permettant la modélisation de sources ponctuelles.

Dans ce chapite, nous allons tout d'abord détailler le processus de
modélisation de scène, puis nous présenterons les résultats d'extraction des
différentes composantes qui la composent. Après avoir montré ces
résultats pour un cas idéal, nous montrerons quelques extractions de cas
plus complexes obtenus avec \hypergal.
\newpage

\section{Modélisation de scène}
% \label{sec:xxx}

\subsection{Projection du cube intrinsèque}
%\label{sec:xxx}

\subsubsection{Seeing relatif Panstarrs/SEDm}
%\label{ssec:xxx}

\subsubsection{Projection spatiale dans l'espace SEDm}
%\label{ssec:xxx}

\subsection{Composantes de la scène}

\subsubsection{Composante du fond: ciel et artefacts}
% \label{ssec:xxx}

\subsubsection{Composante de la galaxie hôte}
%\label{ssec:xxx}

\subsubsection{Composante de la supernova}
% \label{ssec:xxx}

\subsection{Ajustement de la scene}
% \label{ssec:xxx}

\section{Extraction des composantes}

\subsection{Outputs de controle du pipeline}
% \label{ssec:xxx}

\subsection{Extraction de la galaxie hôte}
% \label{ssec:xxx}

\subsection{Extraction de la Supernova}
% \label{ssec:xxx}

\section{Classification: \pkg{SNID}}
% \label{ssec:xxx}

\bibliographystyle{../main/aa_url2}
\bibliography{99_references}

\end{document}

%%% Local Variables:
%%% mode: latex
%%% TeX-master: t
%%% End:
