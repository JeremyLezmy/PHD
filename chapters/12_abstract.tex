\documentclass[../main/main.tex]{subfiles}
\begin{document}


\chapter*{Abstract}
\label{chap:abstract}
\vspace{2cm}
Nearly two decades ago, distances measured from type Ia supernovae 
(SNeIa) were used to discover the accelerated expansion of the
of the universe. While the accelerated expansion of the Universe is now
commonly accepted, understanding what causes this effect is unknown
and requires precise, unbiased measurements. 
Supernovae (SNe) are divided into several categories which can only be
distinguished by their spectral properties, and of all SN types, only the SNeIa
are used as a cosmological probe because of their properties of
standardizable candles. With different types of SNe
having different intrinsic luminosities,
any misclassified objects will lead to a bias in the cosmological
parameters.  

Spectral classification is fairly straightforward for SNe occuring in
isolated environments, but the closer an event occurs close to the core of a
galaxy, the more the spectral contamination intensifies and this analysis
becomes difficult or even impossible.

This PhD thesis has been carried out at the \textit{Institut de Physique des 2
Infinis de Lyon} (IP2I Lyon) as part of the Zwicky Transient Facility (ZTF) wide field cosmological
project, using data obtained from a low-resolution 3D
spectrograph: the Spectral Energy Distribution machine
(SEDm). The objective of this research work is to answer the
problem of the classification of SNe in the case of strong spectral contamination by the host galaxy. 

I present in this manuscript a new method to separate signal from SN
from that of the galaxy in the form
of a scene modeling tool, \hypergal, which allows us to confidently
extract and classify SN previously missed or mis-classified. The core of
this pipeline is based on the use of
photometric data of the host galaxy, taken before the
SN explosion. Using the knowledge of the physics of galaxies, we model
the spectral properties of the host, adjusting and scaling appropriately
to create a 3D cube model of the
isolated host galaxy. Convolving this model to the instrumental response
of SEDm, we produce a projection of this hyperspectral model in the
space of the observations. Modeling the SN as a wavelength dependent
point source, and fitting both the galaxy and supernova simultaneously,
we have produced a scene modelling pipeline that can extract the
properties of the SN from highly contaminated environments. 
This model is validated on simulated samples of SN drawn from observed
data.

At the end of this manuscript, I present the preliminary scientific results from the
ZTF-DR2 dataset of the Type Ia Supernovae $\&$ Cosmology group
of ZTF, composed of $\sim3000$ SNeIa where the vast majority of which has been classified with this
new extraction technique.


\end{document}

%%% Local Variables:
%%% mode: latex
%%% TeX-master: t
%%% End:
