\documentclass[../main/main.tex]{subfiles}
\begin{document}

\chapter{Contexte cosmologique}\label{cp:cosmo}

\epigraph{Et la scène disparaît pour devenir l'actrice}{Person \textsc{Name},
\textit{Doctor Who} S13E08}

\noindent Bien que la cosmologie ne s'en tienne pas aux concepts récents tels
qu'on les connaît et les vulgarise, c'est avec les travaux d'\textsc{Einstein}
au début du XX\ieme~siècle que notre compréhension du monde cosmique prend son
essor. 

%\dominitoc
%\tableofcontents
\minitoc

\begin{enumerate}
    \item RG
    \item Métrique
    \item Constante cosmo
    \item Univers Homogène et isotrope
    \item Courbure
    \item Expansion
    \item Paramètres cosmologiques
    \item Modèle standard
    \item Distance de luminosité
    \item Intérêt SNe
\end{enumerate}
\newpage

\section{Bases de relativité générale}\label{sec:11}

\subsection{Concepts initiaux}\label{ssec:RG}

\subsection{Métrique et équations de conservation}\label{ssec:112}

\subsection{Définition de la constante cosmologique}\label{ssec:lambda}

\section{Introduction du modèle standard de la cosmologie}\label{sec:MS}

\subsection{Univers plat, homogène et isotrope}\label{ssec:plat}

\paragraph*{Univers plat}
\lipsum[1]

\subsection{Métrique de Friedmann-Lemaître-Robertson-Walker}\label{ssec:FLRW}

\subsection{Le modèle \lcdm}\label{ssec:LCDM}

\section{Mesures cosmologiques et distance}\label{sec:dist}

\subsection{Âge de l'Univers}\label{ssec:age}

\subsection{Distance de luminosité}\label{ssec:dl}

\subsection{De la télémétrie aux Supernovae: Calibration}\label{ssec:teltosn}

\section{Supernovae et Cosmologie}\label{sec:snia}

\subsection{Chandelles Standards}
\subsection{Physique de l'explosion}
\subsection{Spectre et classification}\label{ssec:class}
\lipsum[2-4]

\end{document}

%%% Local Variables:
%%% mode: latex
%%% TeX-master: "../main/main"
%%% End:
